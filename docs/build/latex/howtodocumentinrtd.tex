%% Generated by Sphinx.
\def\sphinxdocclass{report}
\documentclass[letterpaper,10pt,english]{sphinxmanual}
\ifdefined\pdfpxdimen
   \let\sphinxpxdimen\pdfpxdimen\else\newdimen\sphinxpxdimen
\fi \sphinxpxdimen=.75bp\relax
\ifdefined\pdfimageresolution
    \pdfimageresolution= \numexpr \dimexpr1in\relax/\sphinxpxdimen\relax
\fi
%% let collapsible pdf bookmarks panel have high depth per default
\PassOptionsToPackage{bookmarksdepth=5}{hyperref}

\PassOptionsToPackage{warn}{textcomp}
\usepackage[utf8]{inputenc}
\ifdefined\DeclareUnicodeCharacter
% support both utf8 and utf8x syntaxes
  \ifdefined\DeclareUnicodeCharacterAsOptional
    \def\sphinxDUC#1{\DeclareUnicodeCharacter{"#1}}
  \else
    \let\sphinxDUC\DeclareUnicodeCharacter
  \fi
  \sphinxDUC{00A0}{\nobreakspace}
  \sphinxDUC{2500}{\sphinxunichar{2500}}
  \sphinxDUC{2502}{\sphinxunichar{2502}}
  \sphinxDUC{2514}{\sphinxunichar{2514}}
  \sphinxDUC{251C}{\sphinxunichar{251C}}
  \sphinxDUC{2572}{\textbackslash}
\fi
\usepackage{cmap}
\usepackage[T1]{fontenc}
\usepackage{amsmath,amssymb,amstext}
\usepackage{babel}



\usepackage{tgtermes}
\usepackage{tgheros}
\renewcommand{\ttdefault}{txtt}



\usepackage[Bjarne]{fncychap}
\usepackage{sphinx}

\fvset{fontsize=auto}
\usepackage{geometry}


% Include hyperref last.
\usepackage{hyperref}
% Fix anchor placement for figures with captions.
\usepackage{hypcap}% it must be loaded after hyperref.
% Set up styles of URL: it should be placed after hyperref.
\urlstyle{same}


\usepackage{sphinxmessages}
\setcounter{tocdepth}{2}


% Jupyter Notebook code cell colors
\definecolor{nbsphinxin}{HTML}{307FC1}
\definecolor{nbsphinxout}{HTML}{BF5B3D}
\definecolor{nbsphinx-code-bg}{HTML}{F5F5F5}
\definecolor{nbsphinx-code-border}{HTML}{E0E0E0}
\definecolor{nbsphinx-stderr}{HTML}{FFDDDD}
% ANSI colors for output streams and traceback highlighting
\definecolor{ansi-black}{HTML}{3E424D}
\definecolor{ansi-black-intense}{HTML}{282C36}
\definecolor{ansi-red}{HTML}{E75C58}
\definecolor{ansi-red-intense}{HTML}{B22B31}
\definecolor{ansi-green}{HTML}{00A250}
\definecolor{ansi-green-intense}{HTML}{007427}
\definecolor{ansi-yellow}{HTML}{DDB62B}
\definecolor{ansi-yellow-intense}{HTML}{B27D12}
\definecolor{ansi-blue}{HTML}{208FFB}
\definecolor{ansi-blue-intense}{HTML}{0065CA}
\definecolor{ansi-magenta}{HTML}{D160C4}
\definecolor{ansi-magenta-intense}{HTML}{A03196}
\definecolor{ansi-cyan}{HTML}{60C6C8}
\definecolor{ansi-cyan-intense}{HTML}{258F8F}
\definecolor{ansi-white}{HTML}{C5C1B4}
\definecolor{ansi-white-intense}{HTML}{A1A6B2}
\definecolor{ansi-default-inverse-fg}{HTML}{FFFFFF}
\definecolor{ansi-default-inverse-bg}{HTML}{000000}

% Define an environment for non-plain-text code cell outputs (e.g. images)
\makeatletter
\newenvironment{nbsphinxfancyoutput}{%
    % Avoid fatal error with framed.sty if graphics too long to fit on one page
    \let\sphinxincludegraphics\nbsphinxincludegraphics
    \nbsphinx@image@maxheight\textheight
    \advance\nbsphinx@image@maxheight -2\fboxsep   % default \fboxsep 3pt
    \advance\nbsphinx@image@maxheight -2\fboxrule  % default \fboxrule 0.4pt
    \advance\nbsphinx@image@maxheight -\baselineskip
\def\nbsphinxfcolorbox{\spx@fcolorbox{nbsphinx-code-border}{white}}%
\def\FrameCommand{\nbsphinxfcolorbox\nbsphinxfancyaddprompt\@empty}%
\def\FirstFrameCommand{\nbsphinxfcolorbox\nbsphinxfancyaddprompt\sphinxVerbatim@Continues}%
\def\MidFrameCommand{\nbsphinxfcolorbox\sphinxVerbatim@Continued\sphinxVerbatim@Continues}%
\def\LastFrameCommand{\nbsphinxfcolorbox\sphinxVerbatim@Continued\@empty}%
\MakeFramed{\advance\hsize-\width\@totalleftmargin\z@\linewidth\hsize\@setminipage}%
\lineskip=1ex\lineskiplimit=1ex\raggedright%
}{\par\unskip\@minipagefalse\endMakeFramed}
\makeatother
\newbox\nbsphinxpromptbox
\def\nbsphinxfancyaddprompt{\ifvoid\nbsphinxpromptbox\else
    \kern\fboxrule\kern\fboxsep
    \copy\nbsphinxpromptbox
    \kern-\ht\nbsphinxpromptbox\kern-\dp\nbsphinxpromptbox
    \kern-\fboxsep\kern-\fboxrule\nointerlineskip
    \fi}
\newlength\nbsphinxcodecellspacing
\setlength{\nbsphinxcodecellspacing}{0pt}

% Define support macros for attaching opening and closing lines to notebooks
\newsavebox\nbsphinxbox
\makeatletter
\newcommand{\nbsphinxstartnotebook}[1]{%
    \par
    % measure needed space
    \setbox\nbsphinxbox\vtop{{#1\par}}
    % reserve some space at bottom of page, else start new page
    \needspace{\dimexpr2.5\baselineskip+\ht\nbsphinxbox+\dp\nbsphinxbox}
    % mimic vertical spacing from \section command
      \addpenalty\@secpenalty
      \@tempskipa 3.5ex \@plus 1ex \@minus .2ex\relax
      \addvspace\@tempskipa
      {\Large\@tempskipa\baselineskip
             \advance\@tempskipa-\prevdepth
             \advance\@tempskipa-\ht\nbsphinxbox
             \ifdim\@tempskipa>\z@
               \vskip \@tempskipa
             \fi}
    \unvbox\nbsphinxbox
    % if notebook starts with a \section, prevent it from adding extra space
    \@nobreaktrue\everypar{\@nobreakfalse\everypar{}}%
    % compensate the parskip which will get inserted by next paragraph
    \nobreak\vskip-\parskip
    % do not break here
    \nobreak
}% end of \nbsphinxstartnotebook

\newcommand{\nbsphinxstopnotebook}[1]{%
    \par
    % measure needed space
    \setbox\nbsphinxbox\vbox{{#1\par}}
    \nobreak % it updates page totals
    \dimen@\pagegoal
    \advance\dimen@-\pagetotal \advance\dimen@-\pagedepth
    \advance\dimen@-\ht\nbsphinxbox \advance\dimen@-\dp\nbsphinxbox
    \ifdim\dimen@<\z@
      % little space left
      \unvbox\nbsphinxbox
      \kern-.8\baselineskip
      \nobreak\vskip\z@\@plus1fil
      \penalty100
      \vskip\z@\@plus-1fil
      \kern.8\baselineskip
    \else
      \unvbox\nbsphinxbox
    \fi
}% end of \nbsphinxstopnotebook

% Ensure height of an included graphics fits in nbsphinxfancyoutput frame
\newdimen\nbsphinx@image@maxheight % set in nbsphinxfancyoutput environment
\newcommand*{\nbsphinxincludegraphics}[2][]{%
    \gdef\spx@includegraphics@options{#1}%
    \setbox\spx@image@box\hbox{\includegraphics[#1,draft]{#2}}%
    \in@false
    \ifdim \wd\spx@image@box>\linewidth
      \g@addto@macro\spx@includegraphics@options{,width=\linewidth}%
      \in@true
    \fi
    % no rotation, no need to worry about depth
    \ifdim \ht\spx@image@box>\nbsphinx@image@maxheight
      \g@addto@macro\spx@includegraphics@options{,height=\nbsphinx@image@maxheight}%
      \in@true
    \fi
    \ifin@
      \g@addto@macro\spx@includegraphics@options{,keepaspectratio}%
    \fi
    \setbox\spx@image@box\box\voidb@x % clear memory
    \expandafter\includegraphics\expandafter[\spx@includegraphics@options]{#2}%
}% end of "\MakeFrame"-safe variant of \sphinxincludegraphics
\makeatother

\makeatletter
\renewcommand*\sphinx@verbatim@nolig@list{\do\'\do\`}
\begingroup
\catcode`'=\active
\let\nbsphinx@noligs\@noligs
\g@addto@macro\nbsphinx@noligs{\let'\PYGZsq}
\endgroup
\makeatother
\renewcommand*\sphinxbreaksbeforeactivelist{\do\<\do\"\do\'}
\renewcommand*\sphinxbreaksafteractivelist{\do\.\do\,\do\:\do\;\do\?\do\!\do\/\do\>\do\-}
\makeatletter
\fvset{codes*=\sphinxbreaksattexescapedchars\do\^\^\let\@noligs\nbsphinx@noligs}
\makeatother



\title{How to Document in RTD}
\date{Jul 20, 2022}
\release{0.0.1}
\author{Bibhash Mitra, Prasang Gupta}
\newcommand{\sphinxlogo}{\vbox{}}
\renewcommand{\releasename}{Release}
\makeindex
\begin{document}

\ifdefined\shorthandoff
  \ifnum\catcode`\=\string=\active\shorthandoff{=}\fi
  \ifnum\catcode`\"=\active\shorthandoff{"}\fi
\fi

\pagestyle{empty}
\sphinxmaketitle
\pagestyle{plain}
\sphinxtableofcontents
\pagestyle{normal}
\phantomsection\label{\detokenize{index::doc}}


\sphinxAtStartPar
This document is a \sphinxstylestrong{templatised} form for a quick documentation of any type of python module.
This serves as a quick documentation solution for a python module / toolkit (\sphinxstyleemphasis{python module} in this case) hosted on Github.
Apart from this, it also serves as a one stop solution for understanding the documentation process in RTD format.

\sphinxAtStartPar
Check out the {\hyperref[\detokenize{installation::doc}]{\sphinxcrossref{\DUrole{doc}{Installation}}}} section for instructions on installation of required packages.

\sphinxAtStartPar
To add a Github\sphinxhyphen{}hosted documentation (in RTD format) to your module / toolkit, check out the {\hyperref[\detokenize{usage::doc}]{\sphinxcrossref{\DUrole{doc}{Usage}}}} section for details and steps.

\sphinxAtStartPar
Feel free to reach out to the authors for any additional information:

\sphinxAtStartPar
\sphinxstyleemphasis{Bibhash C Mitra, Prasang Gupta}

\sphinxstepscope


\chapter{Installation}
\label{\detokenize{installation:installation}}\label{\detokenize{installation::doc}}
\sphinxAtStartPar
Packages required for using this repo can be installed as follows:

\begin{sphinxVerbatim}[commandchars=\\\{\}]
\PYG{g+gp}{\PYGZdl{} }pip install sphinx sphinx\PYGZus{}rtd\PYGZus{}theme sphinx\PYGZhy{}autodoc\PYGZhy{}typehints nbsphinx
\end{sphinxVerbatim}


\sphinxstrong{See also:}
\nopagebreak

\begin{description}
\sphinxlineitem{\sphinxhref{https://www.sphinx-doc.org/en/master/index.html}{Sphinx}}
\sphinxAtStartPar
Understanding sphinx

\sphinxlineitem{\sphinxhref{https://sphinx-rtd-theme.readthedocs.io/en/stable/installing.html}{Sphinx Read the docs theme}}
\sphinxAtStartPar
Understanding how to use the read the docs theme for documentation

\end{description}



\sphinxstepscope


\chapter{Usage}
\label{\detokenize{usage:usage}}\label{\detokenize{usage::doc}}
\sphinxAtStartPar
In this section, we will discuss the implementation details of the documentation.
This includes getting the documentation ready locally and hosting it on github.

\sphinxAtStartPar
We offer 2 solutions for this exercise:
\begin{itemize}
\item {} 
\sphinxAtStartPar
An {\hyperref[\detokenize{usage:in-depth-implementation}]{\sphinxcrossref{\DUrole{std,std-ref}{In\sphinxhyphen{}depth implementation}}}} using \sphinxcode{\sphinxupquote{sphinx\sphinxhyphen{}quickstart}}.
This would include you setting a lot of parameters with provided explanations for your use case.

\sphinxAtStartPar
\sphinxstylestrong{(Recommended if you want to implement and learn at the same time and if you have some spare time on your hands OR if you already are a PRO)}

\item {} 
\sphinxAtStartPar
A {\hyperref[\detokenize{usage:quick-and-dirty-implementation}]{\sphinxcrossref{\DUrole{std,std-ref}{Quick\sphinxhyphen{}and\sphinxhyphen{}Dirty implementation}}}} to get your documentation up and running.
This would involve modifying an already implemented docs code swapping out sections with your case\sphinxhyphen{}specific details.

\sphinxAtStartPar
\sphinxstylestrong{(Recommended if you don’t really care about what is happening in the background OR simply don’t have time)}

\end{itemize}

\sphinxAtStartPar
Before proceeding with any of the steps, make sure that you have installed the necessary packages from the {\hyperref[\detokenize{installation::doc}]{\sphinxcrossref{\DUrole{doc}{Installation}}}} page.


\section{In\sphinxhyphen{}depth implementation}
\label{\detokenize{usage:in-depth-implementation}}

\subsection{Generate basic documentation}
\label{\detokenize{usage:generate-basic-documentation}}\label{\detokenize{usage:generatebasic}}\begin{enumerate}
\sphinxsetlistlabels{\arabic}{enumi}{enumii}{}{.}%
\item {} 
\sphinxAtStartPar
Ensure that your repository is structured as
\begin{quote}

\begin{sphinxVerbatim}[commandchars=\\\{\}]
\PYG{o}{|}\PYG{n}{\PYGZus{}} \PYG{n}{module}
\PYG{o}{|}  \PYG{o}{|}\PYG{n}{\PYGZus{}} \PYG{n+nf+fm}{\PYGZus{}\PYGZus{}init\PYGZus{}\PYGZus{}}\PYG{o}{.}\PYG{n}{py}
\PYG{o}{|}  \PYG{o}{|}\PYG{n}{\PYGZus{}} \PYG{n}{some\PYGZus{}file}\PYG{o}{.}\PYG{n}{py}
\PYG{o}{|}  \PYG{o}{|}\PYG{n}{\PYGZus{}} \PYG{o}{.}\PYG{o}{.}\PYG{o}{.}\PYG{o}{.}
\PYG{o}{|}\PYG{n}{\PYGZus{}} \PYG{n}{setup}\PYG{o}{.}\PYG{n}{py} \PYG{p}{(}\PYG{n}{optional}\PYG{p}{)}
\PYG{o}{|}\PYG{n}{\PYGZus{}} \PYG{n}{README}\PYG{o}{.}\PYG{n}{md}
\PYG{o}{|}\PYG{n}{\PYGZus{}} \PYG{o}{.}\PYG{o}{.}\PYG{o}{.}\PYG{o}{.}
\end{sphinxVerbatim}

\begin{sphinxadmonition}{note}{Note:}
\sphinxAtStartPar
Make sure that \sphinxcode{\sphinxupquote{\_\_init\_\_.py}} file is present in each directory within the main module.
\end{sphinxadmonition}

\begin{sphinxadmonition}{note}{Note:}
\sphinxAtStartPar
The auto\sphinxhyphen{}summary requires proper docstrings to be written. Functions without docstrings would be ignored.
\end{sphinxadmonition}
\end{quote}

\item {} 
\sphinxAtStartPar
Create a folder named \sphinxcode{\sphinxupquote{docs}} in the root directory, so that the structure of repo now is
\begin{quote}

\fvset{hllines={, 1,}}%
\begin{sphinxVerbatim}[commandchars=\\\{\}]
\PYG{o}{|}\PYG{n}{\PYGZus{}} \PYG{n}{docs}
\PYG{o}{|}\PYG{n}{\PYGZus{}} \PYG{n}{module}
\PYG{o}{|}  \PYG{o}{|}\PYG{n}{\PYGZus{}} \PYG{n+nf+fm}{\PYGZus{}\PYGZus{}init\PYGZus{}\PYGZus{}}\PYG{o}{.}\PYG{n}{py}
\PYG{o}{|}  \PYG{o}{|}\PYG{n}{\PYGZus{}} \PYG{n}{some\PYGZus{}file}\PYG{o}{.}\PYG{n}{py}
\PYG{o}{|}  \PYG{o}{|}\PYG{n}{\PYGZus{}} \PYG{o}{.}\PYG{o}{.}\PYG{o}{.}\PYG{o}{.}
\PYG{o}{|}\PYG{n}{\PYGZus{}} \PYG{n}{setup}\PYG{o}{.}\PYG{n}{py} \PYG{p}{(}\PYG{n}{optional}\PYG{p}{)}
\PYG{o}{|}\PYG{n}{\PYGZus{}} \PYG{n}{README}\PYG{o}{.}\PYG{n}{md}
\PYG{o}{|}\PYG{n}{\PYGZus{}} \PYG{o}{.}\PYG{o}{.}\PYG{o}{.}\PYG{o}{.}
\end{sphinxVerbatim}
\sphinxresetverbatimhllines
\end{quote}

\item {} 
\sphinxAtStartPar
Run \sphinxcode{\sphinxupquote{sphinx\sphinxhyphen{}quickstart}} from inside the \sphinxcode{\sphinxupquote{docs}} folder followed by \sphinxcode{\sphinxupquote{make html}}.
This will auto\sphinxhyphen{}generate all necessary files required for a static documentation website.
\begin{quote}

\begin{sphinxVerbatim}[commandchars=\\\{\}]
\PYG{g+gp}{\PYGZdl{} }\PYG{n+nb}{cd} docs
\PYG{g+gp}{\PYGZdl{} }sphinx\PYGZhy{}quickstart

\PYG{g+go}{\PYGZgt{} Separate source and build directories (y/n) [n]: y   \PYGZsh{} This helps in organising the generated files into separate folders}
\PYG{g+go}{\PYGZgt{} Project name: \PYGZlt{}Your Project name goes here\PYGZgt{}}
\PYG{g+go}{\PYGZgt{} Author name(s): \PYGZlt{}Comma separated author names go here\PYGZgt{}}
\PYG{g+go}{\PYGZgt{} Project release []: \PYGZlt{}Version of the module goes here\PYGZgt{}}

\PYG{g+gp}{\PYGZdl{} }make html
\end{sphinxVerbatim}

\sphinxAtStartPar
Files/Folders generated and their functions are listed below
\begin{description}
\sphinxlineitem{build\index{build@\spxentry{build}|spxpagem}\phantomsection\label{\detokenize{usage:term-build}}}
\sphinxAtStartPar
This folder contains HTML files for the static website.

\sphinxlineitem{source\index{source@\spxentry{source}|spxpagem}\phantomsection\label{\detokenize{usage:term-source}}}
\sphinxAtStartPar
This folder contains all the source files needed to build the documentation (rst and configuration files)

\sphinxlineitem{Makefile\index{Makefile@\spxentry{Makefile}|spxpagem}\phantomsection\label{\detokenize{usage:term-Makefile}}}
\sphinxAtStartPar
Makefile for Linux/MacOS and Windows are provided to build the static documentation files (\sphinxstylestrong{build}) from the source files (\sphinxstylestrong{source}).

\end{description}
\end{quote}

\item {} 
\sphinxAtStartPar
At this stage, your documentation would look like this: (You can check by opening \sphinxcode{\sphinxupquote{build/html/index.html}})
\begin{quote}

\begin{figure}[htbp]
\centering

\noindent\sphinxincludegraphics[width=1920\sphinxpxdimen,height=1080\sphinxpxdimen]{{indepth_doc_stage1}.png}
\end{figure}
\end{quote}

\end{enumerate}


\subsection{Modifying configuration file}
\label{\detokenize{usage:modifying-configuration-file}}\begin{enumerate}
\sphinxsetlistlabels{\arabic}{enumi}{enumii}{}{.}%
\item {} 
\sphinxAtStartPar
If you have installed the python module i.e. if you have installed it via \sphinxcode{\sphinxupquote{python setup.py install}} you \sphinxstylestrong{don’t have to do this} but
if you have not installed it then you have to add the following lines to your \sphinxcode{\sphinxupquote{conf.py}} so that it can find your code references:
\begin{quote}

\begin{sphinxVerbatim}[commandchars=\\\{\}]
\PYG{k+kn}{import} \PYG{n+nn}{os}
\PYG{k+kn}{import} \PYG{n+nn}{sys}
\PYG{n}{sys}\PYG{o}{.}\PYG{n}{path}\PYG{o}{.}\PYG{n}{insert}\PYG{p}{(}\PYG{l+m+mi}{0}\PYG{p}{,} \PYG{n}{os}\PYG{o}{.}\PYG{n}{path}\PYG{o}{.}\PYG{n}{abspath}\PYG{p}{(}\PYG{l+s+s1}{\PYGZsq{}}\PYG{l+s+s1}{./../../}\PYG{l+s+s1}{\PYGZsq{}}\PYG{p}{)}\PYG{p}{)}
\end{sphinxVerbatim}

\begin{sphinxadmonition}{note}{Note:}
\sphinxAtStartPar
The abspath needs to be provided for the root of the repository (where the module folder is present) which in our case is \sphinxcode{\sphinxupquote{\textquotesingle{}./../../\textquotesingle{}}}.
This is needed for the autodoc to auto\sphinxhyphen{}generate documentation for the code part.
\end{sphinxadmonition}
\end{quote}

\item {} 
\sphinxAtStartPar
Change the value of copyright variable
\begin{quote}

\begin{sphinxVerbatim}[commandchars=\\\{\}]
\PYG{n}{copyright} \PYG{o}{=} \PYG{l+s+s1}{\PYGZsq{}}\PYG{l+s+s1}{2021, Emerging Technologies}\PYG{l+s+s1}{\PYGZsq{}}
\end{sphinxVerbatim}
\end{quote}

\item {} 
\sphinxAtStartPar
Add the required extensions in the \sphinxcode{\sphinxupquote{extensions}} section
\begin{quote}

\begin{sphinxVerbatim}[commandchars=\\\{\}]
\PYG{n}{extensions} \PYG{o}{=} \PYG{p}{[}
    \PYG{l+s+s1}{\PYGZsq{}}\PYG{l+s+s1}{sphinx.ext.autodoc}\PYG{l+s+s1}{\PYGZsq{}}\PYG{p}{,}           \PYG{c+c1}{\PYGZsh{} for generating documentation from docstrings}
    \PYG{l+s+s1}{\PYGZsq{}}\PYG{l+s+s1}{sphinx.ext.autosummary}\PYG{l+s+s1}{\PYGZsq{}}\PYG{p}{,}       \PYG{c+c1}{\PYGZsh{} generate summaries for autodoc}
    \PYG{l+s+s1}{\PYGZsq{}}\PYG{l+s+s1}{sphinx.ext.autosectionlabel}\PYG{l+s+s1}{\PYGZsq{}}\PYG{p}{,}  \PYG{c+c1}{\PYGZsh{} allow reference sections to use titles}
    \PYG{l+s+s1}{\PYGZsq{}}\PYG{l+s+s1}{sphinx.ext.intersphinx}\PYG{l+s+s1}{\PYGZsq{}}\PYG{p}{,}       \PYG{c+c1}{\PYGZsh{} link to other project documentations}
    \PYG{l+s+s1}{\PYGZsq{}}\PYG{l+s+s1}{sphinx.ext.viewcode}\PYG{l+s+s1}{\PYGZsq{}}\PYG{p}{,}          \PYG{c+c1}{\PYGZsh{} add links to python source code for documentation}
    \PYG{l+s+s1}{\PYGZsq{}}\PYG{l+s+s1}{sphinx\PYGZus{}autodoc\PYGZus{}typehints}\PYG{l+s+s1}{\PYGZsq{}}\PYG{p}{,}     \PYG{c+c1}{\PYGZsh{} automatically document param types}
    \PYG{l+s+s1}{\PYGZsq{}}\PYG{l+s+s1}{nbsphinx}\PYG{l+s+s1}{\PYGZsq{}}\PYG{p}{,}                     \PYG{c+c1}{\PYGZsh{} integrate with jupyter notebooks}
    \PYG{l+s+s1}{\PYGZsq{}}\PYG{l+s+s1}{sphinx.ext.napoleon}\PYG{l+s+s1}{\PYGZsq{}}\PYG{p}{,}          \PYG{c+c1}{\PYGZsh{} support for numpy and google docstrings}
    \PYG{l+s+s1}{\PYGZsq{}}\PYG{l+s+s1}{sphinx.ext.coverage}\PYG{l+s+s1}{\PYGZsq{}}\PYG{p}{,}          \PYG{c+c1}{\PYGZsh{} collect document coverage}
    \PYG{l+s+s1}{\PYGZsq{}}\PYG{l+s+s1}{sphinx\PYGZus{}rtd\PYGZus{}theme}\PYG{l+s+s1}{\PYGZsq{}}              \PYG{c+c1}{\PYGZsh{} RTD theme}
\PYG{p}{]}
\end{sphinxVerbatim}

\begin{sphinxadmonition}{warning}{Warning:}
\sphinxAtStartPar
We have provided here the extensions that would be enough for most purposes.
However, for advanced modifications, you may need to include additional extensions.
\end{sphinxadmonition}

\sphinxAtStartPar
\sphinxhref{https://www.sphinx-doc.org/en/master/usage/extensions/index.html}{Sphinx’s extension page} has more extensions and details about each if you want to get creative !
\end{quote}

\item {} 
\sphinxAtStartPar
Add some variables
\begin{quote}

\begin{sphinxVerbatim}[commandchars=\\\{\}]
\PYG{n}{intersphinx\PYGZus{}mapping} \PYG{o}{=} \PYG{p}{\PYGZob{}}                     \PYG{c+c1}{\PYGZsh{} documents to be included for intersphinx functionality}
    \PYG{l+s+s2}{\PYGZdq{}}\PYG{l+s+s2}{python}\PYG{l+s+s2}{\PYGZdq{}}\PYG{p}{:} \PYG{p}{(}\PYG{l+s+s2}{\PYGZdq{}}\PYG{l+s+s2}{https://docs.python.org/3/}\PYG{l+s+s2}{\PYGZdq{}}\PYG{p}{,} \PYG{k+kc}{None}\PYG{p}{)}\PYG{p}{,}
\PYG{p}{\PYGZcb{}}
\PYG{n}{autosummary\PYGZus{}generate} \PYG{o}{=} \PYG{k+kc}{True}                 \PYG{c+c1}{\PYGZsh{} turn on autosummary generation}
\PYG{n}{autosummary\PYGZus{}generate\PYGZus{}overwrite} \PYG{o}{=} \PYG{k+kc}{True}       \PYG{c+c1}{\PYGZsh{} turn on overwriting for subsequent builds}
\PYG{n}{autoclass\PYGZus{}content} \PYG{o}{=} \PYG{l+s+s2}{\PYGZdq{}}\PYG{l+s+s2}{both}\PYG{l+s+s2}{\PYGZdq{}}                  \PYG{c+c1}{\PYGZsh{} add \PYGZus{}\PYGZus{}init\PYGZus{}\PYGZus{} doc (ie. params) to class summaries}
\PYG{n}{html\PYGZus{}show\PYGZus{}sourcelink} \PYG{o}{=} \PYG{k+kc}{False}                \PYG{c+c1}{\PYGZsh{} remove \PYGZsq{}view source code\PYGZsq{} from top of page (for html, not python)}
\PYG{n}{autodoc\PYGZus{}inherit\PYGZus{}docstrings} \PYG{o}{=} \PYG{k+kc}{True}           \PYG{c+c1}{\PYGZsh{} if no docstring, inherit from base class}
\PYG{n}{set\PYGZus{}type\PYGZus{}checking\PYGZus{}flag} \PYG{o}{=} \PYG{k+kc}{True}               \PYG{c+c1}{\PYGZsh{} enable \PYGZsq{}expensive\PYGZsq{} imports for sphinx\PYGZus{}autodoc\PYGZus{}typehints}
\PYG{n}{nbsphinx\PYGZus{}allow\PYGZus{}errors} \PYG{o}{=} \PYG{k+kc}{True}                \PYG{c+c1}{\PYGZsh{} continue through Jupyter errors}
\PYG{n}{add\PYGZus{}module\PYGZus{}names} \PYG{o}{=} \PYG{k+kc}{False}                    \PYG{c+c1}{\PYGZsh{} remove namespaces from class/method signatures}
\end{sphinxVerbatim}
\end{quote}

\item {} 
\sphinxAtStartPar
Update the theme by replacing the
\begin{quote}

\begin{sphinxVerbatim}[commandchars=\\\{\}]
\PYG{n}{html\PYGZus{}theme} \PYG{o}{=} \PYG{l+s+s1}{\PYGZsq{}}\PYG{l+s+s1}{alabaster}\PYG{l+s+s1}{\PYGZsq{}}
\end{sphinxVerbatim}

\sphinxAtStartPar
line of code with

\begin{sphinxVerbatim}[commandchars=\\\{\}]
\PYG{k}{try}\PYG{p}{:}
    \PYG{k+kn}{import} \PYG{n+nn}{sphinx\PYGZus{}rtd\PYGZus{}theme}
    \PYG{n}{html\PYGZus{}theme} \PYG{o}{=} \PYG{l+s+s2}{\PYGZdq{}}\PYG{l+s+s2}{sphinx\PYGZus{}rtd\PYGZus{}theme}\PYG{l+s+s2}{\PYGZdq{}}
    \PYG{n}{html\PYGZus{}theme\PYGZus{}path} \PYG{o}{=} \PYG{p}{[}\PYG{n}{sphinx\PYGZus{}rtd\PYGZus{}theme}\PYG{o}{.}\PYG{n}{get\PYGZus{}html\PYGZus{}theme\PYGZus{}path}\PYG{p}{(}\PYG{p}{)}\PYG{p}{]}
    \PYG{n}{html\PYGZus{}css\PYGZus{}files} \PYG{o}{=} \PYG{p}{[}\PYG{l+s+s2}{\PYGZdq{}}\PYG{l+s+s2}{readthedocs\PYGZhy{}custom.css}\PYG{l+s+s2}{\PYGZdq{}}\PYG{p}{]}
\PYG{k}{except}\PYG{p}{:}
    \PYG{n}{html\PYGZus{}theme} \PYG{o}{=} \PYG{l+s+s1}{\PYGZsq{}}\PYG{l+s+s1}{alabaster}\PYG{l+s+s1}{\PYGZsq{}}
\end{sphinxVerbatim}

\sphinxAtStartPar
This would render the documentation in the RTD theme.
If however, some error is encountered in loading the theme, it would fall back to the \sphinxcode{\sphinxupquote{alabaster}} theme supported out of the box by Sphinx.
\end{quote}

\item {} 
\sphinxAtStartPar
Update your documentation by running the \sphinxcode{\sphinxupquote{make html}} command again from the \sphinxcode{\sphinxupquote{docs}} directory.
At this stage, your documentation would look like this: (You can check by opening \sphinxcode{\sphinxupquote{build/html/index.html}})
\begin{quote}

\begin{figure}[htbp]
\centering

\noindent\sphinxincludegraphics[width=1920\sphinxpxdimen,height=1080\sphinxpxdimen]{{indepth_doc_stage2}.png}
\end{figure}
\end{quote}

\end{enumerate}


\subsection{Modifying Makefile}
\label{\detokenize{usage:modifying-makefile}}
\sphinxAtStartPar
There are two makefiles which generate after you run \sphinxcode{\sphinxupquote{sphinx\sphinxhyphen{}quickstart}}
\begin{itemize}
\item {} 
\sphinxAtStartPar
\sphinxstylestrong{Makefile}: This makefile is used in Linux or MacOS

\item {} 
\sphinxAtStartPar
\sphinxstylestrong{make.bat}: This makefile is used in Windows

\end{itemize}

\sphinxAtStartPar
So based on which Operating System you are on, you need to modify the corresponding file


\subsubsection{Linux or MacOS}
\label{\detokenize{usage:linux-or-macos}}
\sphinxAtStartPar
In this case we need to replace the line below \sphinxcode{\sphinxupquote{\%: Makefile}} section so that it becomes:
\begin{quote}

\begin{sphinxVerbatim}[commandchars=\\\{\}]
@\PYG{k}{\PYGZdl{}(}SPHINXBUILD\PYG{k}{)} \PYGZhy{}M \PYG{n+nv}{\PYGZdl{}@} \PYG{l+s+s2}{\PYGZdq{}}\PYG{k}{\PYGZdl{}(}SOURCEDIR\PYG{k}{)}\PYG{l+s+s2}{\PYGZdq{}} \PYG{l+s+s2}{\PYGZdq{}}\PYG{k}{\PYGZdl{}(}BUILDDIR\PYG{k}{)}\PYG{l+s+s2}{\PYGZdq{}} \PYG{k}{\PYGZdl{}(}SPHINXOPTS\PYG{k}{)} \PYG{k}{\PYGZdl{}(}O\PYG{k}{)}\PYG{p}{;} touch .nojekyll\PYG{p}{;} \PYG{n+nb}{echo} \PYG{l+s+s1}{\PYGZsq{}\PYGZlt{}meta http\PYGZhy{}equiv=\PYGZdq{}refresh\PYGZdq{} content=\PYGZdq{}0; url=./build/html/index.html\PYGZdq{} /\PYGZgt{}\PYGZsq{}} \PYGZgt{} index.html
\end{sphinxVerbatim}

\begin{sphinxadmonition}{warning}{Warning:}
\sphinxAtStartPar
You might run into an error saying: \sphinxcode{\sphinxupquote{make: *** No rule to make target \textasciigrave{}html\textquotesingle{}.  Stop}}.
If this happens, you need to just correct the indentation and make sure that it starts with a tab rather than spaces.
(This is a GNU make dependency)
\end{sphinxadmonition}
\end{quote}


\subsubsection{Windows}
\label{\detokenize{usage:windows}}
\sphinxAtStartPar
In this case we need to add two lines before \sphinxcode{\sphinxupquote{goto end}} so that it becomes:
\begin{quote}

\begin{sphinxVerbatim}[commandchars=\\\{\}]
\PYG{n+nv}{\PYGZpc{}SPHINXBUILD\PYGZpc{}} \PYGZhy{}M \PYG{n+nv}{\PYGZpc{}1} \PYG{n+nv}{\PYGZpc{}SOURCEDIR\PYGZpc{}} \PYG{n+nv}{\PYGZpc{}BUILDDIR\PYGZpc{}} \PYG{n+nv}{\PYGZpc{}SPHINXOPTS\PYGZpc{}} \PYG{n+nv}{\PYGZpc{}O\PYGZpc{}}
\PYG{k}{type} NUL \PYG{p}{\PYGZgt{}} .nojekyll
\PYG{k}{echo} \PYG{l+s+se}{\PYGZca{}\PYGZlt{}}meta http\PYGZhy{}equiv=\PYG{l+s+s2}{\PYGZdq{}}\PYG{l+s+s2}{refresh}\PYG{l+s+s2}{\PYGZdq{}} content=\PYG{l+s+s2}{\PYGZdq{}}\PYG{l+s+s2}{0; url=./build/html/index.html}\PYG{l+s+s2}{\PYGZdq{}} /\PYG{l+s+se}{\PYGZca{}\PYGZgt{}}  \PYG{p}{\PYGZgt{}} index.html
\PYG{k}{goto} \PYG{n+nl}{end}
\end{sphinxVerbatim}
\end{quote}

\begin{sphinxadmonition}{note}{Note:}
\sphinxAtStartPar
Explanation for adding these two lines:
\begin{itemize}
\item {} 
\sphinxAtStartPar
As we are not using \sphinxstyleemphasis{Jekyll} theme in our project we need to create a \sphinxcode{\sphinxupquote{.nojekyll}} file in the \sphinxcode{\sphinxupquote{/docs}} folder

\item {} 
\sphinxAtStartPar
Now as we have added a \sphinxcode{\sphinxupquote{.nojekyll}} file, \sphinxstyleemphasis{github pages} will try to find a \sphinxcode{\sphinxupquote{.html}} file in the root directory (i.e. \sphinxcode{\sphinxupquote{docs}} which we will select in {\hyperref[\detokenize{usage:github-pages}]{\sphinxcrossref{\DUrole{std,std-ref}{How to host on Github Pages ?}}}}, but our page is present in \sphinxcode{\sphinxupquote{build/html/index.html}}. So we need a helper page which can redirect to the main build page in \sphinxcode{\sphinxupquote{docs}} folder. For that we create a new \sphinxcode{\sphinxupquote{index.html}} file in \sphinxcode{\sphinxupquote{docs}} folder.

\end{itemize}
\end{sphinxadmonition}


\subsection{Adding rst files}
\label{\detokenize{usage:adding-rst-files}}\begin{enumerate}
\sphinxsetlistlabels{\arabic}{enumi}{enumii}{}{.}%
\item {} 
\sphinxAtStartPar
Create a file \sphinxcode{\sphinxupquote{api.rst}} inside the \sphinxcode{\sphinxupquote{source}} directory with the following content.
This file is responsible for creating the autogenerated documentation for the python module.
\begin{quote}

\fvset{hllines={, 9,}}%
\begin{sphinxVerbatim}[commandchars=\\\{\}]
\PYG{n}{API}
\PYG{o}{==}\PYG{o}{==}\PYG{o}{==}

\PYG{o}{.}\PYG{o}{.} \PYG{n}{autosummary}\PYG{p}{:}\PYG{p}{:}
    \PYG{p}{:}\PYG{n}{toctree}\PYG{p}{:} \PYG{n}{\PYGZus{}autosummary}
    \PYG{p}{:}\PYG{n}{template}\PYG{p}{:} \PYG{n}{custom}\PYG{o}{\PYGZhy{}}\PYG{n}{module}\PYG{o}{\PYGZhy{}}\PYG{n}{template}\PYG{o}{.}\PYG{n}{rst}
    \PYG{p}{:}\PYG{n}{recursive}\PYG{p}{:}

    \PYG{n}{module}
\end{sphinxVerbatim}
\sphinxresetverbatimhllines

\begin{sphinxadmonition}{note}{Note:}
\sphinxAtStartPar
Replace \sphinxcode{\sphinxupquote{module}} with the directory name of your python module / toolkit
\end{sphinxadmonition}

\begin{sphinxadmonition}{warning}{Warning:}
\sphinxAtStartPar
We recommend using a template for docstring documentation as default function documentation is not very readable.
You can copy the \sphinxcode{\sphinxupquote{source/\_templates}} folder from the parent repository of this documentation or you can use this \sphinxhref{https://drive.google.com/drive/folders/1w6lrybpvAozJbfDvuCfj2B8vbkhoFPW\_?usp=sharing}{drive link}
to manually download the folder and put it in the \sphinxcode{\sphinxupquote{source}} directory.
If you do not plan on using this template, REMOVE THIS LINE: \sphinxcode{\sphinxupquote{:template: custom\sphinxhyphen{}module\sphinxhyphen{}template.rst}} from the \sphinxcode{\sphinxupquote{api.rst}} file.
\end{sphinxadmonition}
\end{quote}

\item {} 
\sphinxAtStartPar
It is recommended that there are certain sections that are added to the documentation.
We recommend to add the following sections and hence, create an \sphinxcode{\sphinxupquote{.rst}} file for each in the same \sphinxcode{\sphinxupquote{source}} directory.
\begin{quote}
\begin{description}
\sphinxlineitem{\sphinxcode{\sphinxupquote{installation.rst}}\index{installation.rst@\spxentry{installation.rst}|spxpagem}\phantomsection\label{\detokenize{usage:term-installation.rst}}}
\sphinxAtStartPar
Steps to install the dependencies and the package itself with warnings and solutions to common installation problems

\sphinxlineitem{\sphinxcode{\sphinxupquote{changelogs.rst}}\index{changelogs.rst@\spxentry{changelogs.rst}|spxpagem}\phantomsection\label{\detokenize{usage:term-changelogs.rst}}}
\sphinxAtStartPar
Changelogs is an important part of version management to allow rollbacks and efficient debugging

\sphinxlineitem{\sphinxcode{\sphinxupquote{references.rst}}\index{references.rst@\spxentry{references.rst}|spxpagem}\phantomsection\label{\detokenize{usage:term-references.rst}}}
\sphinxAtStartPar
References used for the package development with URLs

\end{description}

\sphinxAtStartPar
Each of these files will be of the form

\begin{sphinxVerbatim}[commandchars=\\\{\}]
HEADING
=======

Content here in appropriate format
\end{sphinxVerbatim}
\end{quote}

\item {} 
\sphinxAtStartPar
Replace the contents of the \sphinxcode{\sphinxupquote{index.rst}} file with the following.
This is done to adhere to a standard style of formatting the main file and section it in a readable format.
\begin{quote}

\begin{sphinxVerbatim}[commandchars=\\\{\}]
.. toctree::
    :hidden:
    :maxdepth: 3

    Home \PYGZlt{}self\PYGZgt{}
    Installation \PYGZlt{}installation\PYGZgt{}
    Module / Toolkit \PYGZlt{}\PYGZus{}autosummary/module\PYGZgt{}
    Changelogs \PYGZlt{}changelogs\PYGZgt{}
    References \PYGZlt{}references\PYGZgt{}

NAME OF THE TOOLKIT
===================

Content to be shown on the main documentation page

...

Authors : Author 1, Author 2, ...
\end{sphinxVerbatim}
\begin{itemize}
\item {} 
\sphinxAtStartPar
This section here starts the Sphinx TOC tree.
The \sphinxcode{\sphinxupquote{maxdepth}} parameter sets the maximum depth for the tree.
\begin{quote}

\begin{sphinxVerbatim}[commandchars=\\\{\}]
.. toctree::
    :hidden:
    :maxdepth: 3
\end{sphinxVerbatim}
\end{quote}

\item {} 
\sphinxAtStartPar
This section lists the entries to populate the left pane of the documentation.
The format followed here is \sphinxcode{\sphinxupquote{XXX \textless{}YYY\textgreater{}}} where \sphinxcode{\sphinxupquote{XXX}} is the name that will be displayed on the left pane
and \sphinxcode{\sphinxupquote{YYY}} is the name of the \sphinxcode{\sphinxupquote{.rst}} file present in the \sphinxcode{\sphinxupquote{source}} directory which will be displayed when a user goes to that section.
\begin{quote}

\begin{sphinxVerbatim}[commandchars=\\\{\}]
Home \PYGZlt{}self\PYGZgt{}
Installation \PYGZlt{}installation\PYGZgt{}
Module / Toolkit \PYGZlt{}\PYGZus{}autosummary/module\PYGZgt{}
Changelogs \PYGZlt{}changelogs\PYGZgt{}
References \PYGZlt{}references\PYGZgt{}
\end{sphinxVerbatim}

\begin{sphinxadmonition}{note}{Note:}
\sphinxAtStartPar
Include any and all sections that need to be added based on the format.
The \sphinxcode{\sphinxupquote{\_autosummary/module}} format is for the autogenerated docstrings documentation where \sphinxcode{\sphinxupquote{module}} needs to be replaced with the name of the python toolkit directory.
\end{sphinxadmonition}
\end{quote}

\item {} 
\sphinxAtStartPar
The rest of the section follows the same structure as the rest of the \sphinxcode{\sphinxupquote{.rst}} files.

\end{itemize}
\end{quote}

\item {} 
\sphinxAtStartPar
Update your documentation by running the \sphinxcode{\sphinxupquote{make html}} command again from the \sphinxcode{\sphinxupquote{docs}} directory.
At this stage, your documentation would look like this: (You can check by opening \sphinxcode{\sphinxupquote{build/html/index.html}})
\begin{quote}

\begin{figure}[htbp]
\centering

\noindent\sphinxincludegraphics[width=1920\sphinxpxdimen,height=1080\sphinxpxdimen]{{indepth_doc_stage3}.png}
\end{figure}
\end{quote}

\end{enumerate}


\section{Quick\sphinxhyphen{}and\sphinxhyphen{}Dirty implementation}
\label{\detokenize{usage:quick-and-dirty-implementation}}

\subsection{How to use the content of this repository directly}
\label{\detokenize{usage:how-to-use-the-content-of-this-repository-directly}}
\sphinxAtStartPar
If you do not care about what is happening in the background and want a very rapid documentation with all the basic components, the following steps will help.
\begin{itemize}
\item {} \begin{description}
\sphinxlineitem{Step 1}
\sphinxAtStartPar
The structure of your repository should have the following structure. For this example we will call this module \sphinxstylestrong{abracadabra}.
\begin{quote}

\sphinxAtStartPar
\sphinxstylestrong{Before}

\begin{sphinxVerbatim}[commandchars=\\\{\}]
\PYG{o}{|}\PYG{n}{\PYGZus{}} \PYG{n}{abracadabra}
\PYG{o}{|}  \PYG{o}{|}\PYG{n}{\PYGZus{}} \PYG{n+nf+fm}{\PYGZus{}\PYGZus{}init\PYGZus{}\PYGZus{}}\PYG{o}{.}\PYG{n}{py}
\PYG{o}{|}  \PYG{o}{|}\PYG{n}{\PYGZus{}} \PYG{n}{some\PYGZus{}file}\PYG{o}{.}\PYG{n}{py}
\PYG{o}{|}  \PYG{o}{|}\PYG{n}{\PYGZus{}} \PYG{o}{.}\PYG{o}{.}\PYG{o}{.}\PYG{o}{.}
\PYG{o}{|}\PYG{n}{\PYGZus{}} \PYG{n}{setup}\PYG{o}{.}\PYG{n}{py} \PYG{p}{(}\PYG{n}{optional}\PYG{p}{)}
\PYG{o}{|}\PYG{n}{\PYGZus{}} \PYG{n}{README}\PYG{o}{.}\PYG{n}{md}
\PYG{o}{|}\PYG{n}{\PYGZus{}} \PYG{o}{.}\PYG{o}{.}\PYG{o}{.}\PYG{o}{.}
\end{sphinxVerbatim}
\end{quote}

\sphinxAtStartPar
Now copy the \sphinxcode{\sphinxupquote{docs}} folder of \sphinxhref{https://github.pwc.com/AIA/howto-read-the-docs}{this repository} in the root directory of your repository containing the python module. So the structure of the repository will become
\begin{quote}

\sphinxAtStartPar
\sphinxstylestrong{After}

\fvset{hllines={, 1,}}%
\begin{sphinxVerbatim}[commandchars=\\\{\}]
\PYG{o}{|}\PYG{n}{\PYGZus{}} \PYG{n}{docs}
\PYG{o}{|}\PYG{n}{\PYGZus{}} \PYG{n}{abracadabra}
\PYG{o}{|}  \PYG{o}{|}\PYG{n}{\PYGZus{}} \PYG{n+nf+fm}{\PYGZus{}\PYGZus{}init\PYGZus{}\PYGZus{}}\PYG{o}{.}\PYG{n}{py}
\PYG{o}{|}  \PYG{o}{|}\PYG{n}{\PYGZus{}} \PYG{n}{some\PYGZus{}file}\PYG{o}{.}\PYG{n}{py}
\PYG{o}{|}  \PYG{o}{|}\PYG{n}{\PYGZus{}} \PYG{o}{.}\PYG{o}{.}\PYG{o}{.}\PYG{o}{.}
\PYG{o}{|}\PYG{n}{\PYGZus{}} \PYG{n}{setup}\PYG{o}{.}\PYG{n}{py} \PYG{p}{(}\PYG{n}{optional}\PYG{p}{)}
\PYG{o}{|}\PYG{n}{\PYGZus{}} \PYG{n}{README}\PYG{o}{.}\PYG{n}{md}
\PYG{o}{|}\PYG{n}{\PYGZus{}} \PYG{o}{.}\PYG{o}{.}\PYG{o}{.}\PYG{o}{.}
\end{sphinxVerbatim}
\sphinxresetverbatimhllines
\end{quote}

\end{description}

\item {} \begin{description}
\sphinxlineitem{Step 2}
\sphinxAtStartPar
Delete all the \sphinxcode{\sphinxupquote{.rst}} files inside \sphinxcode{\sphinxupquote{docs/source/\_autosummary}} as it contains the documentation of the python module of this repository.

\end{description}

\item {} \begin{description}
\sphinxlineitem{Step 3}\begin{itemize}
\item {} 
\sphinxAtStartPar
Inside \sphinxcode{\sphinxupquote{docs/source/conf.py}} you can find a section called \sphinxstylestrong{SECTION TO EDIT} where you can edit the fields according to your project
\begin{quote}
\begin{itemize}
\item {} 
\sphinxAtStartPar
project

\item {} 
\sphinxAtStartPar
copyright

\item {} 
\sphinxAtStartPar
author

\item {} 
\sphinxAtStartPar
release

\end{itemize}

\begin{sphinxadmonition}{warning}{Warning:}
\sphinxAtStartPar
If you have installed the module with \sphinxcode{\sphinxupquote{python setup.py install}} then you need to remove \sphinxcode{\sphinxupquote{sys.path.insert(0, os.path.abspath(\textquotesingle{}./../../\textquotesingle{}))}} from \sphinxcode{\sphinxupquote{docs/source/conf.py}}. This is to ensure there is no conflict between the installed module and the module in this repository.
\end{sphinxadmonition}
\end{quote}

\item {} 
\sphinxAtStartPar
Inside \sphinxcode{\sphinxupquote{docs/source/index.rst}} change:
\begin{itemize}
\item {} 
\sphinxAtStartPar
Content of the introductory section

\item {} 
\sphinxAtStartPar
\sphinxcode{\sphinxupquote{Python module}} to \sphinxcode{\sphinxupquote{abracadabra}}

\item {} 
\sphinxAtStartPar
\sphinxcode{\sphinxupquote{\textless{}\_autosummary/module\textgreater{}}} to \sphinxcode{\sphinxupquote{\textless{}\_autosummary/abracadabra\textgreater{}}}

\end{itemize}

\item {} 
\sphinxAtStartPar
Inside \sphinxcode{\sphinxupquote{docs/source/api.rst}} change:
\begin{itemize}
\item {} 
\sphinxAtStartPar
\sphinxcode{\sphinxupquote{module}} to \sphinxcode{\sphinxupquote{abracadabra}} (i.e. the name of your module or the folder from where autosummary will start generating automated documentations)

\end{itemize}

\item {} 
\sphinxAtStartPar
Inside \sphinxcode{\sphinxupquote{docs/source/installation.rst}} add the steps of your installation.

\item {} 
\sphinxAtStartPar
Inside \sphinxcode{\sphinxupquote{docs/source/usage.rst}} delete everything and add instructions on how to use your repository.
\begin{quote}

\begin{sphinxadmonition}{note}{Note:}
\sphinxAtStartPar
If you do not want any \sphinxstylestrong{Usage} section in your repository, you can remove this \sphinxcode{\sphinxupquote{usage.rst}} file and remove the line \sphinxcode{\sphinxupquote{Usage \textless{}usage\textgreater{}}} from \sphinxcode{\sphinxupquote{index.rst}}.
\end{sphinxadmonition}
\end{quote}

\item {} 
\sphinxAtStartPar
Inside \sphinxcode{\sphinxupquote{docs/source/changelogs.rst}} and \sphinxcode{\sphinxupquote{docs/source/references.rst}} add your changelogs and references respectively.

\end{itemize}

\end{description}

\item {} \begin{description}
\sphinxlineitem{Step 4}\begin{itemize}
\item {} 
\sphinxAtStartPar
Now we need to run the make command to build the html files:
\begin{quote}

\begin{sphinxVerbatim}[commandchars=\\\{\}]
\PYG{g+gp}{\PYGZdl{} }\PYG{n+nb}{cd} docs
\PYG{g+gp}{\PYGZdl{} }make clean  \PYG{c+c1}{\PYGZsh{} This step is to remove all the prebaked files and folders which are not necessary}
\PYG{g+gp}{\PYGZdl{} }make html
\end{sphinxVerbatim}
\end{quote}

\end{itemize}

\end{description}

\end{itemize}

\sphinxAtStartPar
After you complete the following you can see the documentation if you open \sphinxcode{\sphinxupquote{docs/index.html}} in any supported browser.


\section{How to host on Github Pages ?}
\label{\detokenize{usage:how-to-host-on-github-pages}}\label{\detokenize{usage:github-pages}}
\sphinxAtStartPar
To host this page on github pages we have to follow the following steps:
\begin{itemize}
\item {} 
\sphinxAtStartPar
Push all the files to \sphinxcode{\sphinxupquote{github.pwc.com}}
\begin{quote}

\begin{sphinxVerbatim}[commandchars=\\\{\}]
\PYG{g+gp}{\PYGZdl{} }git add docs/*
\PYG{g+gp}{\PYGZdl{} }git add docs/.nojekyll
\PYG{g+gp}{\PYGZdl{} }git commit \PYGZhy{}m \PYGZlt{}message\PYGZgt{}
\PYG{g+gp}{\PYGZdl{} }git push
\end{sphinxVerbatim}
\end{quote}

\item {} 
\sphinxAtStartPar
Go to the \sphinxcode{\sphinxupquote{Settings}} tab on the upper right section of your github repository
\begin{quote}

\begin{figure}[htbp]
\centering

\noindent\sphinxincludegraphics[width=3347\sphinxpxdimen,height=1432\sphinxpxdimen]{{Github_pages_1}.png}
\end{figure}
\end{quote}

\item {} 
\sphinxAtStartPar
Go down to the GitHub Pages section

\item {} 
\sphinxAtStartPar
In the \sphinxcode{\sphinxupquote{Source}} section select the \sphinxcode{\sphinxupquote{branch}} (whichever branch you pushed the files in the previous step) and after that choose the \sphinxcode{\sphinxupquote{/docs}} folder. Then click on \sphinxcode{\sphinxupquote{Save}}

\item {} 
\sphinxAtStartPar
Now an URL will be generated with a note \sphinxcode{\sphinxupquote{Your site is published at \textless{}URL\textgreater{}}}, Click on the same.
\begin{quote}

\begin{figure}[htbp]
\centering

\noindent\sphinxincludegraphics[width=1866\sphinxpxdimen,height=1060\sphinxpxdimen]{{Github_pages_2}.png}
\end{figure}
\end{quote}

\item {} 
\sphinxAtStartPar
Voila ! Your documentation is ready.
\begin{quote}

\begin{figure}[htbp]
\centering

\noindent\sphinxincludegraphics[width=3353\sphinxpxdimen,height=1882\sphinxpxdimen]{{Github_pages_3}.png}
\end{figure}
\end{quote}

\end{itemize}


\section{How to Document?}
\label{\detokenize{usage:how-to-document}}
\sphinxAtStartPar
In this section we have consolidated all the most commonly used sphinx syntaxes used in various parts of documentation. You can choose a snippet and add it in your documentation.


\subsection{Inline formattings}
\label{\detokenize{usage:inline-formattings}}\label{\detokenize{usage:inline-formatting}}\begin{itemize}
\item {} 
\sphinxAtStartPar
one asterisk: \sphinxcode{\sphinxupquote{*text*}} for Italic. \sphinxstyleemphasis{text}

\item {} 
\sphinxAtStartPar
two asterisks: \sphinxcode{\sphinxupquote{**text**}} for Bold. \sphinxstylestrong{text}

\item {} 
\sphinxAtStartPar
backquotes: \sphinxcode{\sphinxupquote{\textasciigrave{}\textasciigrave{}text\textasciigrave{}\textasciigrave{}}} for code samples. \sphinxcode{\sphinxupquote{text}}

\end{itemize}


\subsection{Numbered and Bullet list}
\label{\detokenize{usage:numbered-and-bullet-list}}\begin{enumerate}
\sphinxsetlistlabels{\arabic}{enumi}{enumii}{}{.}%
\item {} 
\sphinxAtStartPar
This is a numbered list.

\item {} 
\sphinxAtStartPar
It has two items too.

\item {} 
\sphinxAtStartPar
This is a numbered list using \sphinxcode{\sphinxupquote{\#}}

\end{enumerate}
\begin{itemize}
\item {} 
\sphinxAtStartPar
First bullet

\item {} 
\sphinxAtStartPar
Second bullet
\begin{itemize}
\item {} 
\sphinxAtStartPar
A nested list with a space above

\item {} 
\sphinxAtStartPar
and some subitems

\end{itemize}

\item {} \begin{description}
\sphinxlineitem{Third bullet}\begin{itemize}
\item {} 
\sphinxAtStartPar
Nested list without space

\end{itemize}

\end{description}

\end{itemize}


\subsection{Hyperlinks}
\label{\detokenize{usage:hyperlinks}}
\sphinxAtStartPar
This is an \sphinxhref{https://www.sphinx-doc.org/en/master/usage/restructuredtext/basics.html}{hyperlink}

\sphinxAtStartPar
This is an internal reference {\hyperref[\detokenize{usage:inline-formatting}]{\sphinxcrossref{\DUrole{std,std-ref}{Inline formattings}}}}


\subsection{Code blocks}
\label{\detokenize{usage:code-blocks}}
\sphinxAtStartPar
Adding \sphinxcode{\sphinxupquote{::}} at the end of paragraph and indenting them with \sphinxstyleemphasis{4 spaces} results in a code block as below

\begin{sphinxVerbatim}[commandchars=\\\{\}]
\PYG{n}{Code} \PYG{n}{block} \PYG{n}{line} \PYG{l+m+mi}{1}
\PYG{n}{Code} \PYG{n}{block} \PYG{n}{line} \PYG{l+m+mi}{2}
\end{sphinxVerbatim}


\subsection{Tables}
\label{\detokenize{usage:tables}}
\sphinxAtStartPar
Grid Table
\begin{quote}


\begin{savenotes}\sphinxattablestart
\centering
\begin{tabulary}{\linewidth}[t]{|T|T|T|T|}
\hline
\sphinxstyletheadfamily 
\sphinxAtStartPar
Header row, column 1
(header rows optional)
&\sphinxstyletheadfamily 
\sphinxAtStartPar
Header 2
&\sphinxstyletheadfamily 
\sphinxAtStartPar
Header 3
&\sphinxstyletheadfamily 
\sphinxAtStartPar
Header 4
\\
\hline
\sphinxAtStartPar
body row 1, column 1
&
\sphinxAtStartPar
column 2
&
\sphinxAtStartPar
column 3
&
\sphinxAtStartPar
column 4
\\
\hline
\sphinxAtStartPar
body row 2
&
\sphinxAtStartPar
…
&
\sphinxAtStartPar
…
&\\
\hline
\end{tabulary}
\par
\sphinxattableend\end{savenotes}
\end{quote}

\sphinxAtStartPar
Simple Table
\begin{quote}


\begin{savenotes}\sphinxattablestart
\centering
\begin{tabulary}{\linewidth}[t]{|T|T|T|}
\hline
\sphinxstyletheadfamily 
\sphinxAtStartPar
A
&\sphinxstyletheadfamily 
\sphinxAtStartPar
B
&\sphinxstyletheadfamily 
\sphinxAtStartPar
A and B
\\
\hline
\sphinxAtStartPar
False
&
\sphinxAtStartPar
False
&
\sphinxAtStartPar
False
\\
\hline
\sphinxAtStartPar
True
&
\sphinxAtStartPar
False
&
\sphinxAtStartPar
False
\\
\hline
\sphinxAtStartPar
False
&
\sphinxAtStartPar
True
&
\sphinxAtStartPar
False
\\
\hline
\sphinxAtStartPar
True
&
\sphinxAtStartPar
True
&
\sphinxAtStartPar
True
\\
\hline
\end{tabulary}
\par
\sphinxattableend\end{savenotes}
\end{quote}


\subsection{Sections}
\label{\detokenize{usage:sections}}
\sphinxAtStartPar
This is a heading:

\begin{sphinxVerbatim}[commandchars=\\\{\}]
\PYG{n}{This} \PYG{o+ow}{is} \PYG{n}{a} \PYG{n}{Heading}
\PYG{o}{==}\PYG{o}{==}\PYG{o}{==}\PYG{o}{==}\PYG{o}{==}\PYG{o}{==}\PYG{o}{==}\PYG{o}{==}\PYG{o}{=}
\end{sphinxVerbatim}


\section{This is a subheading}
\label{\detokenize{usage:this-is-a-subheading}}

\subsection{This is a subsubheading}
\label{\detokenize{usage:this-is-a-subsubheading}}

\subsubsection{This is a paragraph}
\label{\detokenize{usage:this-is-a-paragraph}}

\paragraph{This is a part}
\label{\detokenize{usage:this-is-a-part}}

\subparagraph{This is a chapter}
\label{\detokenize{usage:this-is-a-chapter}}
\begin{sphinxadmonition}{note}{Note:}
\sphinxAtStartPar
If a Heading is defined it will add a section in the side bar, so it is enclosed in a block in the above example
\end{sphinxadmonition}


\subsection{Roles}
\label{\detokenize{usage:roles}}\begin{itemize}
\item {} 
\sphinxAtStartPar
\sphinxstyleemphasis{emphasis}

\item {} 
\sphinxAtStartPar
\sphinxstylestrong{strong}

\item {} 
\sphinxAtStartPar
\sphinxcode{\sphinxupquote{literal}}

\item {} 
\sphinxAtStartPar
$_{\text{subscript text}}$

\item {} 
\sphinxAtStartPar
$^{\text{superscript text}}$

\item {} 
\sphinxAtStartPar
\sphinxtitleref{for titles of books, periodicals, and other materials}

\item {} 
\sphinxAtStartPar
\(a^2 + b^2 = c^2\).

\item {} 
\sphinxAtStartPar
\sphinxmenuselection{Start \(\rightarrow\) Programs}

\end{itemize}

\sphinxAtStartPar
For more \sphinxstylestrong{roles} please visit \sphinxhref{https://www.sphinx-doc.org/en/master/usage/restructuredtext/roles.html}{here}


\subsection{Directives}
\label{\detokenize{usage:directives}}
\sphinxAtStartPar
This is a figure.

\begin{figure}[htbp]
\centering
\capstart

\sphinxhref{https://www.google.com/search?q=deep+learning+\&tbm=isch\&ved=2ahUKEwj24NyX1K70AhUmD1kFHS2LCH4Q2-cCegQIABAA\&oq=deep+learning+\&gs\_lcp=CgNpbWcQAzIFCAAQgAQyBQgAEIAEMgUIABCABDIFCAAQgAQyBQgAEIAEMgUIABCABDIFCAAQgAQyBQgAEIAEMgUIABCABDIFCAAQgAQ6CAgAELEDEIMBOggIABCABBCxAzoECAAQAzoECAAQQzoHCAAQsQMQQzoLCAAQgAQQsQMQgwFQ1gdY6BhguiJoAHAAeACAAesCiAGxJZIBBDMtMTWYAQCgAQGqAQtnd3Mtd2l6LWltZ8ABAQ\&sclient=img\&ei=oPWcYfaQGqae5NoPrZai8Ac\&bih=722\&biw=1536\&rlz=1C1GCEA\_enIN979IN979\#imgrc=EpN1KteHpbsUhM}{\sphinxincludegraphics[width=640\sphinxpxdimen,height=360\sphinxpxdimen]{{Sample}.jpg}}
\caption{This is the caption}\label{\detokenize{usage:id2}}
\begin{sphinxlegend}
\sphinxAtStartPar
This is a legend
\end{sphinxlegend}
\end{figure}

\sphinxAtStartPar
This is a code block

\begin{sphinxVerbatim}[commandchars=\\\{\}]
\PYG{k}{def} \PYG{n+nf}{my\PYGZus{}function}\PYG{p}{(}\PYG{p}{)}\PYG{p}{:}
    \PYG{l+s+s2}{\PYGZdq{}}\PYG{l+s+s2}{Loren Ipsum}\PYG{l+s+s2}{\PYGZdq{}}
    \PYG{n+nb}{print}\PYG{p}{(}\PYG{l+s+s2}{\PYGZdq{}}\PYG{l+s+s2}{Hello World}\PYG{l+s+s2}{\PYGZdq{}}\PYG{p}{)}
\end{sphinxVerbatim}

\sphinxAtStartPar
This is a math block
\begin{equation*}
\begin{split}α_t(i) = P(O_1, O_2, … O_t, q_t = S_i λ)\end{split}
\end{equation*}
\sphinxAtStartPar
This is a note

\begin{sphinxadmonition}{note}{Note:}
\sphinxAtStartPar
This is a note
\end{sphinxadmonition}

\sphinxAtStartPar
This is a warning

\begin{sphinxadmonition}{warning}{Warning:}
\sphinxAtStartPar
This is a warning
\end{sphinxadmonition}
\begin{itemize}
\item {} 
\sphinxAtStartPar
This is to add a version
\begin{quote}

\sphinxAtStartPar
\DUrole{versionmodified,added}{New in version 2.5: }The version is 2.5
\end{quote}

\item {} 
\sphinxAtStartPar
This is to note down version changes
\begin{quote}

\sphinxAtStartPar
\DUrole{versionmodified,changed}{Changed in version 2.6: }The version changed
\end{quote}

\item {} 
\sphinxAtStartPar
This is to show deprecated functions
\begin{quote}

\sphinxAtStartPar
\DUrole{versionmodified,deprecated}{Deprecated since version 3.1: }Use \sphinxcode{\sphinxupquote{demo()}} instead.
\end{quote}

\item {} 
\sphinxAtStartPar
This is a glossary
\begin{quote}
\begin{description}
\sphinxlineitem{Term 1\index{Term 1@\spxentry{Term 1}|spxpagem}\phantomsection\label{\detokenize{usage:term-Term-1}}}
\sphinxAtStartPar
Definition of term 1.

\sphinxlineitem{Term 2\index{Term 2@\spxentry{Term 2}|spxpagem}\phantomsection\label{\detokenize{usage:term-Term-2}}}
\sphinxAtStartPar
Definition of term 2.

\end{description}
\end{quote}

\end{itemize}

\sphinxAtStartPar
For more \sphinxstylestrong{directives} please visit \sphinxhref{https://www.sphinx-doc.org/en/master/usage/restructuredtext/directives.html}{here}

\sphinxstepscope


\chapter{module}
\label{\detokenize{_autosummary/module:module-module}}\label{\detokenize{_autosummary/module:module}}\label{\detokenize{_autosummary/module::doc}}\index{module@\spxentry{module}!module@\spxentry{module}}\index{module@\spxentry{module}!module@\spxentry{module}}

\begin{savenotes}\sphinxattablestart
\centering
\begin{tabulary}{\linewidth}[t]{\X{1}{2}\X{1}{2}}
\hline

\sphinxAtStartPar
{\hyperref[\detokenize{_autosummary/module.dummycode:module-module.dummycode}]{\sphinxcrossref{\sphinxcode{\sphinxupquote{module.dummycode}}}}}
&
\sphinxAtStartPar
This script contains some dummy codes with their respective docstrings to use them for templatisation
\\
\hline
\sphinxAtStartPar
{\hyperref[\detokenize{_autosummary/module.submodule:module-module.submodule}]{\sphinxcrossref{\sphinxcode{\sphinxupquote{module.submodule}}}}}
&
\sphinxAtStartPar

\\
\hline
\end{tabulary}
\par
\sphinxattableend\end{savenotes}

\sphinxstepscope


\section{module.dummycode}
\label{\detokenize{_autosummary/module.dummycode:module-module.dummycode}}\label{\detokenize{_autosummary/module.dummycode:module-dummycode}}\label{\detokenize{_autosummary/module.dummycode::doc}}\index{module@\spxentry{module}!module.dummycode@\spxentry{module.dummycode}}\index{module.dummycode@\spxentry{module.dummycode}!module@\spxentry{module}}
\sphinxAtStartPar
This script contains some dummy codes with their respective docstrings
to use them for templatisation
\subsubsection*{Functions}


\begin{savenotes}\sphinxattablestart
\centering
\begin{tabulary}{\linewidth}[t]{\X{1}{2}\X{1}{2}}
\hline

\sphinxAtStartPar
{\hyperref[\detokenize{_autosummary/module.dummycode.some_function:module.dummycode.some_function}]{\sphinxcrossref{\sphinxcode{\sphinxupquote{some\_function}}}}}
&
\sphinxAtStartPar
Summary or Description of the Function
\\
\hline
\sphinxAtStartPar
{\hyperref[\detokenize{_autosummary/module.dummycode.square:module.dummycode.square}]{\sphinxcrossref{\sphinxcode{\sphinxupquote{square}}}}}
&
\sphinxAtStartPar
Returned argument a is squared.
\\
\hline
\sphinxAtStartPar
{\hyperref[\detokenize{_autosummary/module.dummycode.string_reverse:module.dummycode.string_reverse}]{\sphinxcrossref{\sphinxcode{\sphinxupquote{string\_reverse}}}}}
&
\sphinxAtStartPar
Returns the reversed String.
\\
\hline
\end{tabulary}
\par
\sphinxattableend\end{savenotes}

\sphinxstepscope


\subsection{module.dummycode.some\_function}
\label{\detokenize{_autosummary/module.dummycode.some_function:module-dummycode-some-function}}\label{\detokenize{_autosummary/module.dummycode.some_function::doc}}\index{some\_function() (in module module.dummycode)@\spxentry{some\_function()}\spxextra{in module module.dummycode}}

\begin{fulllineitems}
\phantomsection\label{\detokenize{_autosummary/module.dummycode.some_function:module.dummycode.some_function}}
\pysigstartsignatures
\pysiglinewithargsret{\sphinxbfcode{\sphinxupquote{some\_function}}}{\emph{\DUrole{n}{argument1}}}{}
\pysigstopsignatures
\sphinxAtStartPar
Summary or Description of the Function
\begin{quote}\begin{description}
\sphinxlineitem{Parameters}
\sphinxAtStartPar
\sphinxstyleliteralstrong{\sphinxupquote{argument1}} (\sphinxhref{https://docs.python.org/3/library/functions.html\#int}{\sphinxstyleliteralemphasis{\sphinxupquote{int}}}) \textendash{} Description of arg1

\sphinxlineitem{Returns}
\sphinxAtStartPar
Returning value

\sphinxlineitem{Return type}
\sphinxAtStartPar
\sphinxhref{https://docs.python.org/3/library/functions.html\#int}{int}

\end{description}\end{quote}

\end{fulllineitems}


\sphinxstepscope


\subsection{module.dummycode.square}
\label{\detokenize{_autosummary/module.dummycode.square:module-dummycode-square}}\label{\detokenize{_autosummary/module.dummycode.square::doc}}\index{square() (in module module.dummycode)@\spxentry{square()}\spxextra{in module module.dummycode}}

\begin{fulllineitems}
\phantomsection\label{\detokenize{_autosummary/module.dummycode.square:module.dummycode.square}}
\pysigstartsignatures
\pysiglinewithargsret{\sphinxbfcode{\sphinxupquote{square}}}{\emph{\DUrole{n}{a}}}{}
\pysigstopsignatures
\sphinxAtStartPar
Returned argument a is squared.

\end{fulllineitems}


\sphinxstepscope


\subsection{module.dummycode.string\_reverse}
\label{\detokenize{_autosummary/module.dummycode.string_reverse:module-dummycode-string-reverse}}\label{\detokenize{_autosummary/module.dummycode.string_reverse::doc}}\index{string\_reverse() (in module module.dummycode)@\spxentry{string\_reverse()}\spxextra{in module module.dummycode}}

\begin{fulllineitems}
\phantomsection\label{\detokenize{_autosummary/module.dummycode.string_reverse:module.dummycode.string_reverse}}
\pysigstartsignatures
\pysiglinewithargsret{\sphinxbfcode{\sphinxupquote{string\_reverse}}}{\emph{\DUrole{n}{str1}}}{}
\pysigstopsignatures
\sphinxAtStartPar
Returns the reversed String.
\begin{quote}\begin{description}
\sphinxlineitem{Parameters}
\sphinxAtStartPar
\sphinxstyleliteralstrong{\sphinxupquote{str1}} (\sphinxhref{https://docs.python.org/3/library/stdtypes.html\#str}{\sphinxstyleliteralemphasis{\sphinxupquote{str}}}) \textendash{} The string which is to be reversed.

\sphinxlineitem{Returns}
\sphinxAtStartPar
The string which gets reversed.

\sphinxlineitem{Return type}
\sphinxAtStartPar
reverse(str1)

\end{description}\end{quote}

\end{fulllineitems}

\subsubsection*{Classes}


\begin{savenotes}\sphinxattablestart
\centering
\begin{tabulary}{\linewidth}[t]{\X{1}{2}\X{1}{2}}
\hline

\sphinxAtStartPar
{\hyperref[\detokenize{_autosummary/module.dummycode.GoogleVehicle:module.dummycode.GoogleVehicle}]{\sphinxcrossref{\sphinxcode{\sphinxupquote{GoogleVehicle}}}}}
&
\sphinxAtStartPar
The Vehicle object contains a lot of vehicles
\\
\hline
\sphinxAtStartPar
{\hyperref[\detokenize{_autosummary/module.dummycode.NumpyVehicle:module.dummycode.NumpyVehicle}]{\sphinxcrossref{\sphinxcode{\sphinxupquote{NumpyVehicle}}}}}
&
\sphinxAtStartPar
The Vehicles object contains lots of vehicles
\\
\hline
\sphinxAtStartPar
{\hyperref[\detokenize{_autosummary/module.dummycode.SphinxVehicle:module.dummycode.SphinxVehicle}]{\sphinxcrossref{\sphinxcode{\sphinxupquote{SphinxVehicle}}}}}
&
\sphinxAtStartPar
The Vehicle object contains lots of vehicles
\\
\hline
\end{tabulary}
\par
\sphinxattableend\end{savenotes}

\sphinxstepscope


\subsection{module.dummycode.GoogleVehicle}
\label{\detokenize{_autosummary/module.dummycode.GoogleVehicle:module-dummycode-googlevehicle}}\label{\detokenize{_autosummary/module.dummycode.GoogleVehicle::doc}}\index{GoogleVehicle (class in module.dummycode)@\spxentry{GoogleVehicle}\spxextra{class in module.dummycode}}

\begin{fulllineitems}
\phantomsection\label{\detokenize{_autosummary/module.dummycode.GoogleVehicle:module.dummycode.GoogleVehicle}}
\pysigstartsignatures
\pysiglinewithargsret{\sphinxbfcode{\sphinxupquote{class\DUrole{w}{  }}}\sphinxbfcode{\sphinxupquote{GoogleVehicle}}}{\emph{\DUrole{n}{arg}}, \emph{\DUrole{o}{*}\DUrole{n}{args}}, \emph{\DUrole{o}{**}\DUrole{n}{kwargs}}}{}
\pysigstopsignatures
\sphinxAtStartPar
Bases: \sphinxhref{https://docs.python.org/3/library/functions.html\#object}{\sphinxcode{\sphinxupquote{object}}}

\sphinxAtStartPar
The Vehicle object contains a lot of vehicles
\begin{quote}\begin{description}
\sphinxlineitem{Parameters}\begin{itemize}
\item {} 
\sphinxAtStartPar
\sphinxstyleliteralstrong{\sphinxupquote{arg}} (\sphinxhref{https://docs.python.org/3/library/stdtypes.html\#str}{\sphinxstyleliteralemphasis{\sphinxupquote{str}}}) \textendash{} The arg is used for…

\item {} 
\sphinxAtStartPar
\sphinxstyleliteralstrong{\sphinxupquote{*args}} \textendash{} The variable arguments are used for…

\item {} 
\sphinxAtStartPar
\sphinxstyleliteralstrong{\sphinxupquote{**kwargs}} \textendash{} The keyword arguments are used for…

\end{itemize}

\end{description}\end{quote}
\index{arg (GoogleVehicle attribute)@\spxentry{arg}\spxextra{GoogleVehicle attribute}}

\begin{fulllineitems}
\phantomsection\label{\detokenize{_autosummary/module.dummycode.GoogleVehicle:module.dummycode.GoogleVehicle.arg}}
\pysigstartsignatures
\pysigline{\sphinxbfcode{\sphinxupquote{arg}}}
\pysigstopsignatures
\sphinxAtStartPar
This is where we store arg,
\begin{quote}\begin{description}
\sphinxlineitem{Type}
\sphinxAtStartPar
\sphinxhref{https://docs.python.org/3/library/stdtypes.html\#str}{str}

\end{description}\end{quote}

\end{fulllineitems}

\subsubsection*{Methods}


\begin{savenotes}\sphinxattablestart
\centering
\begin{tabulary}{\linewidth}[t]{\X{1}{2}\X{1}{2}}
\hline

\sphinxAtStartPar
{\hyperref[\detokenize{_autosummary/module.dummycode.GoogleVehicle:module.dummycode.GoogleVehicle.cars}]{\sphinxcrossref{\sphinxcode{\sphinxupquote{cars}}}}}
&
\sphinxAtStartPar
We can\textquotesingle{}t travel distance in vehicles without fuels, so here is the fuels
\\
\hline
\end{tabulary}
\par
\sphinxattableend\end{savenotes}
\index{cars() (GoogleVehicle method)@\spxentry{cars()}\spxextra{GoogleVehicle method}}

\begin{fulllineitems}
\phantomsection\label{\detokenize{_autosummary/module.dummycode.GoogleVehicle:module.dummycode.GoogleVehicle.cars}}
\pysigstartsignatures
\pysiglinewithargsret{\sphinxbfcode{\sphinxupquote{cars}}}{\emph{\DUrole{n}{distance}}, \emph{\DUrole{n}{destination}}}{}
\pysigstopsignatures
\sphinxAtStartPar
We can’t travel distance in vehicles without fuels, so here is the fuels
\begin{quote}\begin{description}
\sphinxlineitem{Parameters}\begin{itemize}
\item {} 
\sphinxAtStartPar
\sphinxstyleliteralstrong{\sphinxupquote{distance}} (\sphinxhref{https://docs.python.org/3/library/functions.html\#int}{\sphinxstyleliteralemphasis{\sphinxupquote{int}}}) \textendash{} The amount of distance traveled

\item {} 
\sphinxAtStartPar
\sphinxstyleliteralstrong{\sphinxupquote{destination}} (\sphinxhref{https://docs.python.org/3/library/functions.html\#bool}{\sphinxstyleliteralemphasis{\sphinxupquote{bool}}}) \textendash{} Should the fuels refilled to cover the distance?

\end{itemize}

\sphinxlineitem{Raises}
\sphinxAtStartPar
\sphinxhref{https://docs.python.org/3/library/exceptions.html\#RuntimeError}{\sphinxstyleliteralstrong{\sphinxupquote{RuntimeError}}} \textendash{} Out of fuel

\sphinxlineitem{Returns}
\sphinxAtStartPar
A car mileage

\sphinxlineitem{Return type}
\sphinxAtStartPar
cars

\end{description}\end{quote}

\end{fulllineitems}


\end{fulllineitems}


\sphinxstepscope


\subsection{module.dummycode.NumpyVehicle}
\label{\detokenize{_autosummary/module.dummycode.NumpyVehicle:module-dummycode-numpyvehicle}}\label{\detokenize{_autosummary/module.dummycode.NumpyVehicle::doc}}\index{NumpyVehicle (class in module.dummycode)@\spxentry{NumpyVehicle}\spxextra{class in module.dummycode}}

\begin{fulllineitems}
\phantomsection\label{\detokenize{_autosummary/module.dummycode.NumpyVehicle:module.dummycode.NumpyVehicle}}
\pysigstartsignatures
\pysiglinewithargsret{\sphinxbfcode{\sphinxupquote{class\DUrole{w}{  }}}\sphinxbfcode{\sphinxupquote{NumpyVehicle}}}{\emph{\DUrole{n}{arg}}, \emph{\DUrole{o}{*}\DUrole{n}{args}}, \emph{\DUrole{o}{**}\DUrole{n}{kwargs}}}{}
\pysigstopsignatures
\sphinxAtStartPar
Bases: \sphinxhref{https://docs.python.org/3/library/functions.html\#object}{\sphinxcode{\sphinxupquote{object}}}

\sphinxAtStartPar
The Vehicles object contains lots of vehicles
\begin{quote}\begin{description}
\sphinxlineitem{Parameters}\begin{itemize}
\item {} 
\sphinxAtStartPar
\sphinxstyleliteralstrong{\sphinxupquote{arg}} (\sphinxhref{https://docs.python.org/3/library/stdtypes.html\#str}{\sphinxstyleliteralemphasis{\sphinxupquote{str}}}) \textendash{} The arg is used for …

\item {} 
\sphinxAtStartPar
\sphinxstyleliteralstrong{\sphinxupquote{*args}} \textendash{} The variable arguments are used for …

\item {} 
\sphinxAtStartPar
\sphinxstyleliteralstrong{\sphinxupquote{**kwargs}} \textendash{} The keyword arguments are used for …

\end{itemize}

\end{description}\end{quote}
\index{arg (NumpyVehicle attribute)@\spxentry{arg}\spxextra{NumpyVehicle attribute}}

\begin{fulllineitems}
\phantomsection\label{\detokenize{_autosummary/module.dummycode.NumpyVehicle:module.dummycode.NumpyVehicle.arg}}
\pysigstartsignatures
\pysigline{\sphinxbfcode{\sphinxupquote{arg}}}
\pysigstopsignatures
\sphinxAtStartPar
This is where we store arg,
\begin{quote}\begin{description}
\sphinxlineitem{Type}
\sphinxAtStartPar
\sphinxhref{https://docs.python.org/3/library/stdtypes.html\#str}{str}

\end{description}\end{quote}

\end{fulllineitems}

\subsubsection*{Methods}


\begin{savenotes}\sphinxattablestart
\centering
\begin{tabulary}{\linewidth}[t]{\X{1}{2}\X{1}{2}}
\hline

\sphinxAtStartPar
{\hyperref[\detokenize{_autosummary/module.dummycode.NumpyVehicle:module.dummycode.NumpyVehicle.cars}]{\sphinxcrossref{\sphinxcode{\sphinxupquote{cars}}}}}
&
\sphinxAtStartPar
We can\textquotesingle{}t travel distance in vehicles without fuels, so here is the fuels
\\
\hline
\end{tabulary}
\par
\sphinxattableend\end{savenotes}
\index{cars() (NumpyVehicle method)@\spxentry{cars()}\spxextra{NumpyVehicle method}}

\begin{fulllineitems}
\phantomsection\label{\detokenize{_autosummary/module.dummycode.NumpyVehicle:module.dummycode.NumpyVehicle.cars}}
\pysigstartsignatures
\pysiglinewithargsret{\sphinxbfcode{\sphinxupquote{cars}}}{\emph{\DUrole{n}{distance}}, \emph{\DUrole{n}{destination}}}{}
\pysigstopsignatures
\sphinxAtStartPar
We can’t travel distance in vehicles without fuels, so here is the fuels
\begin{quote}\begin{description}
\sphinxlineitem{Parameters}\begin{itemize}
\item {} 
\sphinxAtStartPar
\sphinxstyleliteralstrong{\sphinxupquote{distance}} (\sphinxhref{https://docs.python.org/3/library/functions.html\#int}{\sphinxstyleliteralemphasis{\sphinxupquote{int}}}) \textendash{} The amount of distance traveled

\item {} 
\sphinxAtStartPar
\sphinxstyleliteralstrong{\sphinxupquote{destination}} (\sphinxhref{https://docs.python.org/3/library/functions.html\#bool}{\sphinxstyleliteralemphasis{\sphinxupquote{bool}}}) \textendash{} Should the fuels refilled to cover the distance?

\end{itemize}

\sphinxlineitem{Raises}
\sphinxAtStartPar
\sphinxhref{https://docs.python.org/3/library/exceptions.html\#RuntimeError}{\sphinxstyleliteralstrong{\sphinxupquote{RuntimeError}}} \textendash{} Out of fuel

\sphinxlineitem{Returns}
\sphinxAtStartPar
A car mileage

\sphinxlineitem{Return type}
\sphinxAtStartPar
cars

\end{description}\end{quote}

\end{fulllineitems}


\end{fulllineitems}


\sphinxstepscope


\subsection{module.dummycode.SphinxVehicle}
\label{\detokenize{_autosummary/module.dummycode.SphinxVehicle:module-dummycode-sphinxvehicle}}\label{\detokenize{_autosummary/module.dummycode.SphinxVehicle::doc}}\index{SphinxVehicle (class in module.dummycode)@\spxentry{SphinxVehicle}\spxextra{class in module.dummycode}}

\begin{fulllineitems}
\phantomsection\label{\detokenize{_autosummary/module.dummycode.SphinxVehicle:module.dummycode.SphinxVehicle}}
\pysigstartsignatures
\pysiglinewithargsret{\sphinxbfcode{\sphinxupquote{class\DUrole{w}{  }}}\sphinxbfcode{\sphinxupquote{SphinxVehicle}}}{\emph{\DUrole{n}{arg}}, \emph{\DUrole{o}{*}\DUrole{n}{args}}, \emph{\DUrole{o}{**}\DUrole{n}{kwargs}}}{}
\pysigstopsignatures
\sphinxAtStartPar
Bases: \sphinxhref{https://docs.python.org/3/library/functions.html\#object}{\sphinxcode{\sphinxupquote{object}}}

\sphinxAtStartPar
The Vehicle object contains lots of vehicles
\begin{quote}\begin{description}
\sphinxlineitem{Parameters}\begin{itemize}
\item {} 
\sphinxAtStartPar
\sphinxstyleliteralstrong{\sphinxupquote{arg}} (\sphinxhref{https://docs.python.org/3/library/stdtypes.html\#str}{\sphinxstyleliteralemphasis{\sphinxupquote{str}}}) \textendash{} The arg is used for …

\item {} 
\sphinxAtStartPar
\sphinxstyleliteralstrong{\sphinxupquote{*args}} \textendash{} The variable arguments are used for …

\item {} 
\sphinxAtStartPar
\sphinxstyleliteralstrong{\sphinxupquote{**kwargs}} \textendash{} The keyword arguments are used for …

\end{itemize}

\sphinxlineitem{Variables}
\sphinxAtStartPar
\sphinxstyleliteralstrong{\sphinxupquote{arg}} (\sphinxhref{https://docs.python.org/3/library/stdtypes.html\#str}{\sphinxstyleliteralemphasis{\sphinxupquote{str}}}) \textendash{} This is where we store arg

\end{description}\end{quote}
\subsubsection*{Methods}


\begin{savenotes}\sphinxattablestart
\centering
\begin{tabulary}{\linewidth}[t]{\X{1}{2}\X{1}{2}}
\hline

\sphinxAtStartPar
{\hyperref[\detokenize{_autosummary/module.dummycode.SphinxVehicle:module.dummycode.SphinxVehicle.cars}]{\sphinxcrossref{\sphinxcode{\sphinxupquote{cars}}}}}
&
\sphinxAtStartPar
We can\textquotesingle{}t travel a certain distance in vehicles without fuels, so here\textquotesingle{}s the fuels
\\
\hline
\end{tabulary}
\par
\sphinxattableend\end{savenotes}
\index{cars() (SphinxVehicle method)@\spxentry{cars()}\spxextra{SphinxVehicle method}}

\begin{fulllineitems}
\phantomsection\label{\detokenize{_autosummary/module.dummycode.SphinxVehicle:module.dummycode.SphinxVehicle.cars}}
\pysigstartsignatures
\pysiglinewithargsret{\sphinxbfcode{\sphinxupquote{cars}}}{\emph{\DUrole{n}{distance}}, \emph{\DUrole{n}{destination}}}{}
\pysigstopsignatures
\sphinxAtStartPar
We can’t travel a certain distance in vehicles without fuels, so here’s the fuels
\begin{quote}\begin{description}
\sphinxlineitem{Parameters}\begin{itemize}
\item {} 
\sphinxAtStartPar
\sphinxstyleliteralstrong{\sphinxupquote{distance}} \textendash{} The amount of distance traveled

\item {} 
\sphinxAtStartPar
\sphinxstyleliteralstrong{\sphinxupquote{destinationReached}} (\sphinxhref{https://docs.python.org/3/library/functions.html\#bool}{\sphinxstyleliteralemphasis{\sphinxupquote{bool}}}) \textendash{} Should the fuels be refilled to cover required distance?

\end{itemize}

\sphinxlineitem{Raises}
\sphinxAtStartPar
\sphinxhref{https://docs.python.org/3/library/exceptions.html\#RuntimeError}{\sphinxcode{\sphinxupquote{RuntimeError}}}: Out of fuel

\sphinxlineitem{Returns}
\sphinxAtStartPar
A Car mileage

\sphinxlineitem{Return type}
\sphinxAtStartPar
Cars

\end{description}\end{quote}

\end{fulllineitems}


\end{fulllineitems}


\sphinxstepscope


\section{module.submodule}
\label{\detokenize{_autosummary/module.submodule:module-module.submodule}}\label{\detokenize{_autosummary/module.submodule:module-submodule}}\label{\detokenize{_autosummary/module.submodule::doc}}\index{module@\spxentry{module}!module.submodule@\spxentry{module.submodule}}\index{module.submodule@\spxentry{module.submodule}!module@\spxentry{module}}

\begin{savenotes}\sphinxattablestart
\centering
\begin{tabulary}{\linewidth}[t]{\X{1}{2}\X{1}{2}}
\hline

\sphinxAtStartPar
{\hyperref[\detokenize{_autosummary/module.submodule.dummycode:module-module.submodule.dummycode}]{\sphinxcrossref{\sphinxcode{\sphinxupquote{module.submodule.dummycode}}}}}
&
\sphinxAtStartPar
This script contains some dummy codes with their respective docstrings to use them for templatisation
\\
\hline
\end{tabulary}
\par
\sphinxattableend\end{savenotes}

\sphinxstepscope


\subsection{module.submodule.dummycode}
\label{\detokenize{_autosummary/module.submodule.dummycode:module-module.submodule.dummycode}}\label{\detokenize{_autosummary/module.submodule.dummycode:module-submodule-dummycode}}\label{\detokenize{_autosummary/module.submodule.dummycode::doc}}\index{module@\spxentry{module}!module.submodule.dummycode@\spxentry{module.submodule.dummycode}}\index{module.submodule.dummycode@\spxentry{module.submodule.dummycode}!module@\spxentry{module}}
\sphinxAtStartPar
This script contains some dummy codes with their respective docstrings
to use them for templatisation
\subsubsection*{Functions}


\begin{savenotes}\sphinxattablestart
\centering
\begin{tabulary}{\linewidth}[t]{\X{1}{2}\X{1}{2}}
\hline

\sphinxAtStartPar
{\hyperref[\detokenize{_autosummary/module.submodule.dummycode.multiply:module.submodule.dummycode.multiply}]{\sphinxcrossref{\sphinxcode{\sphinxupquote{multiply}}}}}
&
\sphinxAtStartPar
Multiply two numbers with each other
\\
\hline
\sphinxAtStartPar
{\hyperref[\detokenize{_autosummary/module.submodule.dummycode.some_function:module.submodule.dummycode.some_function}]{\sphinxcrossref{\sphinxcode{\sphinxupquote{some\_function}}}}}
&
\sphinxAtStartPar
Summary or Description of the Function
\\
\hline
\sphinxAtStartPar
{\hyperref[\detokenize{_autosummary/module.submodule.dummycode.string_reverse:module.submodule.dummycode.string_reverse}]{\sphinxcrossref{\sphinxcode{\sphinxupquote{string\_reverse}}}}}
&
\sphinxAtStartPar
Returns the reversed String.
\\
\hline
\end{tabulary}
\par
\sphinxattableend\end{savenotes}

\sphinxstepscope


\subsubsection{module.submodule.dummycode.multiply}
\label{\detokenize{_autosummary/module.submodule.dummycode.multiply:module-submodule-dummycode-multiply}}\label{\detokenize{_autosummary/module.submodule.dummycode.multiply::doc}}\index{multiply() (in module module.submodule.dummycode)@\spxentry{multiply()}\spxextra{in module module.submodule.dummycode}}

\begin{fulllineitems}
\phantomsection\label{\detokenize{_autosummary/module.submodule.dummycode.multiply:module.submodule.dummycode.multiply}}
\pysigstartsignatures
\pysiglinewithargsret{\sphinxbfcode{\sphinxupquote{multiply}}}{\emph{\DUrole{n}{a}}, \emph{\DUrole{n}{b}}}{}
\pysigstopsignatures
\sphinxAtStartPar
Multiply two numbers with each other
\begin{quote}\begin{description}
\sphinxlineitem{Parameters}\begin{itemize}
\item {} 
\sphinxAtStartPar
\sphinxstyleliteralstrong{\sphinxupquote{a}} (\sphinxhref{https://docs.python.org/3/library/functions.html\#int}{\sphinxstyleliteralemphasis{\sphinxupquote{int}}}) \textendash{} first number

\item {} 
\sphinxAtStartPar
\sphinxstyleliteralstrong{\sphinxupquote{b}} (\sphinxhref{https://docs.python.org/3/library/functions.html\#int}{\sphinxstyleliteralemphasis{\sphinxupquote{int}}}) \textendash{} second number

\end{itemize}

\sphinxlineitem{Returns}
\sphinxAtStartPar
a x b

\sphinxlineitem{Return type}
\sphinxAtStartPar
\sphinxhref{https://docs.python.org/3/library/functions.html\#int}{int}

\end{description}\end{quote}

\end{fulllineitems}


\sphinxstepscope


\subsubsection{module.submodule.dummycode.some\_function}
\label{\detokenize{_autosummary/module.submodule.dummycode.some_function:module-submodule-dummycode-some-function}}\label{\detokenize{_autosummary/module.submodule.dummycode.some_function::doc}}\index{some\_function() (in module module.submodule.dummycode)@\spxentry{some\_function()}\spxextra{in module module.submodule.dummycode}}

\begin{fulllineitems}
\phantomsection\label{\detokenize{_autosummary/module.submodule.dummycode.some_function:module.submodule.dummycode.some_function}}
\pysigstartsignatures
\pysiglinewithargsret{\sphinxbfcode{\sphinxupquote{some\_function}}}{\emph{\DUrole{n}{argument1}}}{}
\pysigstopsignatures
\sphinxAtStartPar
Summary or Description of the Function
\begin{quote}\begin{description}
\sphinxlineitem{Parameters}
\sphinxAtStartPar
\sphinxstyleliteralstrong{\sphinxupquote{argument1}} (\sphinxhref{https://docs.python.org/3/library/functions.html\#int}{\sphinxstyleliteralemphasis{\sphinxupquote{int}}}) \textendash{} Description of arg1

\sphinxlineitem{Returns}
\sphinxAtStartPar
Returning value

\sphinxlineitem{Return type}
\sphinxAtStartPar
\sphinxhref{https://docs.python.org/3/library/functions.html\#int}{int}

\end{description}\end{quote}

\end{fulllineitems}


\sphinxstepscope


\subsubsection{module.submodule.dummycode.string\_reverse}
\label{\detokenize{_autosummary/module.submodule.dummycode.string_reverse:module-submodule-dummycode-string-reverse}}\label{\detokenize{_autosummary/module.submodule.dummycode.string_reverse::doc}}\index{string\_reverse() (in module module.submodule.dummycode)@\spxentry{string\_reverse()}\spxextra{in module module.submodule.dummycode}}

\begin{fulllineitems}
\phantomsection\label{\detokenize{_autosummary/module.submodule.dummycode.string_reverse:module.submodule.dummycode.string_reverse}}
\pysigstartsignatures
\pysiglinewithargsret{\sphinxbfcode{\sphinxupquote{string\_reverse}}}{\emph{\DUrole{n}{str1}}}{}
\pysigstopsignatures
\sphinxAtStartPar
Returns the reversed String.
\begin{quote}\begin{description}
\sphinxlineitem{Parameters}
\sphinxAtStartPar
\sphinxstyleliteralstrong{\sphinxupquote{str1}} (\sphinxhref{https://docs.python.org/3/library/stdtypes.html\#str}{\sphinxstyleliteralemphasis{\sphinxupquote{str}}}) \textendash{} The string which is to be reversed.

\sphinxlineitem{Returns}
\sphinxAtStartPar
The string which gets reversed.

\sphinxlineitem{Return type}
\sphinxAtStartPar
reverse(str1)

\end{description}\end{quote}

\end{fulllineitems}

\subsubsection*{Classes}


\begin{savenotes}\sphinxattablestart
\centering
\begin{tabulary}{\linewidth}[t]{\X{1}{2}\X{1}{2}}
\hline

\sphinxAtStartPar
{\hyperref[\detokenize{_autosummary/module.submodule.dummycode.GoogleVehicle:module.submodule.dummycode.GoogleVehicle}]{\sphinxcrossref{\sphinxcode{\sphinxupquote{GoogleVehicle}}}}}
&
\sphinxAtStartPar
The Vehicle object contains a lot of vehicles
\\
\hline
\sphinxAtStartPar
{\hyperref[\detokenize{_autosummary/module.submodule.dummycode.NumpyVehicle:module.submodule.dummycode.NumpyVehicle}]{\sphinxcrossref{\sphinxcode{\sphinxupquote{NumpyVehicle}}}}}
&
\sphinxAtStartPar
The Vehicles object contains lots of vehicles
\\
\hline
\sphinxAtStartPar
{\hyperref[\detokenize{_autosummary/module.submodule.dummycode.SphinxVehicle:module.submodule.dummycode.SphinxVehicle}]{\sphinxcrossref{\sphinxcode{\sphinxupquote{SphinxVehicle}}}}}
&
\sphinxAtStartPar
The Vehicle object contains lots of vehicles
\\
\hline
\end{tabulary}
\par
\sphinxattableend\end{savenotes}

\sphinxstepscope


\subsubsection{module.submodule.dummycode.GoogleVehicle}
\label{\detokenize{_autosummary/module.submodule.dummycode.GoogleVehicle:module-submodule-dummycode-googlevehicle}}\label{\detokenize{_autosummary/module.submodule.dummycode.GoogleVehicle::doc}}\index{GoogleVehicle (class in module.submodule.dummycode)@\spxentry{GoogleVehicle}\spxextra{class in module.submodule.dummycode}}

\begin{fulllineitems}
\phantomsection\label{\detokenize{_autosummary/module.submodule.dummycode.GoogleVehicle:module.submodule.dummycode.GoogleVehicle}}
\pysigstartsignatures
\pysiglinewithargsret{\sphinxbfcode{\sphinxupquote{class\DUrole{w}{  }}}\sphinxbfcode{\sphinxupquote{GoogleVehicle}}}{\emph{\DUrole{n}{arg}}, \emph{\DUrole{o}{*}\DUrole{n}{args}}, \emph{\DUrole{o}{**}\DUrole{n}{kwargs}}}{}
\pysigstopsignatures
\sphinxAtStartPar
Bases: \sphinxhref{https://docs.python.org/3/library/functions.html\#object}{\sphinxcode{\sphinxupquote{object}}}

\sphinxAtStartPar
The Vehicle object contains a lot of vehicles
\begin{quote}\begin{description}
\sphinxlineitem{Parameters}\begin{itemize}
\item {} 
\sphinxAtStartPar
\sphinxstyleliteralstrong{\sphinxupquote{arg}} (\sphinxhref{https://docs.python.org/3/library/stdtypes.html\#str}{\sphinxstyleliteralemphasis{\sphinxupquote{str}}}) \textendash{} The arg is used for…

\item {} 
\sphinxAtStartPar
\sphinxstyleliteralstrong{\sphinxupquote{*args}} \textendash{} The variable arguments are used for…

\item {} 
\sphinxAtStartPar
\sphinxstyleliteralstrong{\sphinxupquote{**kwargs}} \textendash{} The keyword arguments are used for…

\end{itemize}

\end{description}\end{quote}
\index{arg (GoogleVehicle attribute)@\spxentry{arg}\spxextra{GoogleVehicle attribute}}

\begin{fulllineitems}
\phantomsection\label{\detokenize{_autosummary/module.submodule.dummycode.GoogleVehicle:module.submodule.dummycode.GoogleVehicle.arg}}
\pysigstartsignatures
\pysigline{\sphinxbfcode{\sphinxupquote{arg}}}
\pysigstopsignatures
\sphinxAtStartPar
This is where we store arg,
\begin{quote}\begin{description}
\sphinxlineitem{Type}
\sphinxAtStartPar
\sphinxhref{https://docs.python.org/3/library/stdtypes.html\#str}{str}

\end{description}\end{quote}

\end{fulllineitems}

\subsubsection*{Methods}


\begin{savenotes}\sphinxattablestart
\centering
\begin{tabulary}{\linewidth}[t]{\X{1}{2}\X{1}{2}}
\hline

\sphinxAtStartPar
{\hyperref[\detokenize{_autosummary/module.submodule.dummycode.GoogleVehicle:module.submodule.dummycode.GoogleVehicle.cars}]{\sphinxcrossref{\sphinxcode{\sphinxupquote{cars}}}}}
&
\sphinxAtStartPar
We can\textquotesingle{}t travel distance in vehicles without fuels, so here is the fuels
\\
\hline
\end{tabulary}
\par
\sphinxattableend\end{savenotes}
\index{cars() (GoogleVehicle method)@\spxentry{cars()}\spxextra{GoogleVehicle method}}

\begin{fulllineitems}
\phantomsection\label{\detokenize{_autosummary/module.submodule.dummycode.GoogleVehicle:module.submodule.dummycode.GoogleVehicle.cars}}
\pysigstartsignatures
\pysiglinewithargsret{\sphinxbfcode{\sphinxupquote{cars}}}{\emph{\DUrole{n}{distance}}, \emph{\DUrole{n}{destination}}}{}
\pysigstopsignatures
\sphinxAtStartPar
We can’t travel distance in vehicles without fuels, so here is the fuels
\begin{quote}\begin{description}
\sphinxlineitem{Parameters}\begin{itemize}
\item {} 
\sphinxAtStartPar
\sphinxstyleliteralstrong{\sphinxupquote{distance}} (\sphinxhref{https://docs.python.org/3/library/functions.html\#int}{\sphinxstyleliteralemphasis{\sphinxupquote{int}}}) \textendash{} The amount of distance traveled

\item {} 
\sphinxAtStartPar
\sphinxstyleliteralstrong{\sphinxupquote{destination}} (\sphinxhref{https://docs.python.org/3/library/functions.html\#bool}{\sphinxstyleliteralemphasis{\sphinxupquote{bool}}}) \textendash{} Should the fuels refilled to cover the distance?

\end{itemize}

\sphinxlineitem{Raises}
\sphinxAtStartPar
\sphinxhref{https://docs.python.org/3/library/exceptions.html\#RuntimeError}{\sphinxstyleliteralstrong{\sphinxupquote{RuntimeError}}} \textendash{} Out of fuel

\sphinxlineitem{Returns}
\sphinxAtStartPar
A car mileage

\sphinxlineitem{Return type}
\sphinxAtStartPar
cars

\end{description}\end{quote}

\end{fulllineitems}


\end{fulllineitems}


\sphinxstepscope


\subsubsection{module.submodule.dummycode.NumpyVehicle}
\label{\detokenize{_autosummary/module.submodule.dummycode.NumpyVehicle:module-submodule-dummycode-numpyvehicle}}\label{\detokenize{_autosummary/module.submodule.dummycode.NumpyVehicle::doc}}\index{NumpyVehicle (class in module.submodule.dummycode)@\spxentry{NumpyVehicle}\spxextra{class in module.submodule.dummycode}}

\begin{fulllineitems}
\phantomsection\label{\detokenize{_autosummary/module.submodule.dummycode.NumpyVehicle:module.submodule.dummycode.NumpyVehicle}}
\pysigstartsignatures
\pysiglinewithargsret{\sphinxbfcode{\sphinxupquote{class\DUrole{w}{  }}}\sphinxbfcode{\sphinxupquote{NumpyVehicle}}}{\emph{\DUrole{n}{arg}}, \emph{\DUrole{o}{*}\DUrole{n}{args}}, \emph{\DUrole{o}{**}\DUrole{n}{kwargs}}}{}
\pysigstopsignatures
\sphinxAtStartPar
Bases: \sphinxhref{https://docs.python.org/3/library/functions.html\#object}{\sphinxcode{\sphinxupquote{object}}}

\sphinxAtStartPar
The Vehicles object contains lots of vehicles
\begin{quote}\begin{description}
\sphinxlineitem{Parameters}\begin{itemize}
\item {} 
\sphinxAtStartPar
\sphinxstyleliteralstrong{\sphinxupquote{arg}} (\sphinxhref{https://docs.python.org/3/library/stdtypes.html\#str}{\sphinxstyleliteralemphasis{\sphinxupquote{str}}}) \textendash{} The arg is used for …

\item {} 
\sphinxAtStartPar
\sphinxstyleliteralstrong{\sphinxupquote{*args}} \textendash{} The variable arguments are used for …

\item {} 
\sphinxAtStartPar
\sphinxstyleliteralstrong{\sphinxupquote{**kwargs}} \textendash{} The keyword arguments are used for …

\end{itemize}

\end{description}\end{quote}
\index{arg (NumpyVehicle attribute)@\spxentry{arg}\spxextra{NumpyVehicle attribute}}

\begin{fulllineitems}
\phantomsection\label{\detokenize{_autosummary/module.submodule.dummycode.NumpyVehicle:module.submodule.dummycode.NumpyVehicle.arg}}
\pysigstartsignatures
\pysigline{\sphinxbfcode{\sphinxupquote{arg}}}
\pysigstopsignatures
\sphinxAtStartPar
This is where we store arg,
\begin{quote}\begin{description}
\sphinxlineitem{Type}
\sphinxAtStartPar
\sphinxhref{https://docs.python.org/3/library/stdtypes.html\#str}{str}

\end{description}\end{quote}

\end{fulllineitems}

\subsubsection*{Methods}


\begin{savenotes}\sphinxattablestart
\centering
\begin{tabulary}{\linewidth}[t]{\X{1}{2}\X{1}{2}}
\hline

\sphinxAtStartPar
{\hyperref[\detokenize{_autosummary/module.submodule.dummycode.NumpyVehicle:module.submodule.dummycode.NumpyVehicle.cars}]{\sphinxcrossref{\sphinxcode{\sphinxupquote{cars}}}}}
&
\sphinxAtStartPar
We can\textquotesingle{}t travel distance in vehicles without fuels, so here is the fuels
\\
\hline
\end{tabulary}
\par
\sphinxattableend\end{savenotes}
\index{cars() (NumpyVehicle method)@\spxentry{cars()}\spxextra{NumpyVehicle method}}

\begin{fulllineitems}
\phantomsection\label{\detokenize{_autosummary/module.submodule.dummycode.NumpyVehicle:module.submodule.dummycode.NumpyVehicle.cars}}
\pysigstartsignatures
\pysiglinewithargsret{\sphinxbfcode{\sphinxupquote{cars}}}{\emph{\DUrole{n}{distance}}, \emph{\DUrole{n}{destination}}}{}
\pysigstopsignatures
\sphinxAtStartPar
We can’t travel distance in vehicles without fuels, so here is the fuels
\begin{quote}\begin{description}
\sphinxlineitem{Parameters}\begin{itemize}
\item {} 
\sphinxAtStartPar
\sphinxstyleliteralstrong{\sphinxupquote{distance}} (\sphinxhref{https://docs.python.org/3/library/functions.html\#int}{\sphinxstyleliteralemphasis{\sphinxupquote{int}}}) \textendash{} The amount of distance traveled

\item {} 
\sphinxAtStartPar
\sphinxstyleliteralstrong{\sphinxupquote{destination}} (\sphinxhref{https://docs.python.org/3/library/functions.html\#bool}{\sphinxstyleliteralemphasis{\sphinxupquote{bool}}}) \textendash{} Should the fuels refilled to cover the distance?

\end{itemize}

\sphinxlineitem{Raises}
\sphinxAtStartPar
\sphinxhref{https://docs.python.org/3/library/exceptions.html\#RuntimeError}{\sphinxstyleliteralstrong{\sphinxupquote{RuntimeError}}} \textendash{} Out of fuel

\sphinxlineitem{Returns}
\sphinxAtStartPar
A car mileage

\sphinxlineitem{Return type}
\sphinxAtStartPar
cars

\end{description}\end{quote}

\end{fulllineitems}


\end{fulllineitems}


\sphinxstepscope


\subsubsection{module.submodule.dummycode.SphinxVehicle}
\label{\detokenize{_autosummary/module.submodule.dummycode.SphinxVehicle:module-submodule-dummycode-sphinxvehicle}}\label{\detokenize{_autosummary/module.submodule.dummycode.SphinxVehicle::doc}}\index{SphinxVehicle (class in module.submodule.dummycode)@\spxentry{SphinxVehicle}\spxextra{class in module.submodule.dummycode}}

\begin{fulllineitems}
\phantomsection\label{\detokenize{_autosummary/module.submodule.dummycode.SphinxVehicle:module.submodule.dummycode.SphinxVehicle}}
\pysigstartsignatures
\pysiglinewithargsret{\sphinxbfcode{\sphinxupquote{class\DUrole{w}{  }}}\sphinxbfcode{\sphinxupquote{SphinxVehicle}}}{\emph{\DUrole{n}{arg}}, \emph{\DUrole{o}{*}\DUrole{n}{args}}, \emph{\DUrole{o}{**}\DUrole{n}{kwargs}}}{}
\pysigstopsignatures
\sphinxAtStartPar
Bases: \sphinxhref{https://docs.python.org/3/library/functions.html\#object}{\sphinxcode{\sphinxupquote{object}}}

\sphinxAtStartPar
The Vehicle object contains lots of vehicles
\begin{quote}\begin{description}
\sphinxlineitem{Parameters}\begin{itemize}
\item {} 
\sphinxAtStartPar
\sphinxstyleliteralstrong{\sphinxupquote{arg}} (\sphinxhref{https://docs.python.org/3/library/stdtypes.html\#str}{\sphinxstyleliteralemphasis{\sphinxupquote{str}}}) \textendash{} The arg is used for …

\item {} 
\sphinxAtStartPar
\sphinxstyleliteralstrong{\sphinxupquote{*args}} \textendash{} The variable arguments are used for …

\item {} 
\sphinxAtStartPar
\sphinxstyleliteralstrong{\sphinxupquote{**kwargs}} \textendash{} The keyword arguments are used for …

\end{itemize}

\sphinxlineitem{Variables}
\sphinxAtStartPar
\sphinxstyleliteralstrong{\sphinxupquote{arg}} (\sphinxhref{https://docs.python.org/3/library/stdtypes.html\#str}{\sphinxstyleliteralemphasis{\sphinxupquote{str}}}) \textendash{} This is where we store arg

\end{description}\end{quote}
\subsubsection*{Methods}


\begin{savenotes}\sphinxattablestart
\centering
\begin{tabulary}{\linewidth}[t]{\X{1}{2}\X{1}{2}}
\hline

\sphinxAtStartPar
{\hyperref[\detokenize{_autosummary/module.submodule.dummycode.SphinxVehicle:module.submodule.dummycode.SphinxVehicle.cars}]{\sphinxcrossref{\sphinxcode{\sphinxupquote{cars}}}}}
&
\sphinxAtStartPar
We can\textquotesingle{}t travel a certain distance in vehicles without fuels, so here\textquotesingle{}s the fuels
\\
\hline
\end{tabulary}
\par
\sphinxattableend\end{savenotes}
\index{cars() (SphinxVehicle method)@\spxentry{cars()}\spxextra{SphinxVehicle method}}

\begin{fulllineitems}
\phantomsection\label{\detokenize{_autosummary/module.submodule.dummycode.SphinxVehicle:module.submodule.dummycode.SphinxVehicle.cars}}
\pysigstartsignatures
\pysiglinewithargsret{\sphinxbfcode{\sphinxupquote{cars}}}{\emph{\DUrole{n}{distance}}, \emph{\DUrole{n}{destination}}}{}
\pysigstopsignatures
\sphinxAtStartPar
We can’t travel a certain distance in vehicles without fuels, so here’s the fuels
\begin{quote}\begin{description}
\sphinxlineitem{Parameters}\begin{itemize}
\item {} 
\sphinxAtStartPar
\sphinxstyleliteralstrong{\sphinxupquote{distance}} \textendash{} The amount of distance traveled

\item {} 
\sphinxAtStartPar
\sphinxstyleliteralstrong{\sphinxupquote{destinationReached}} (\sphinxhref{https://docs.python.org/3/library/functions.html\#bool}{\sphinxstyleliteralemphasis{\sphinxupquote{bool}}}) \textendash{} Should the fuels be refilled to cover required distance?

\end{itemize}

\sphinxlineitem{Raises}
\sphinxAtStartPar
\sphinxhref{https://docs.python.org/3/library/exceptions.html\#RuntimeError}{\sphinxcode{\sphinxupquote{RuntimeError}}}: Out of fuel

\sphinxlineitem{Returns}
\sphinxAtStartPar
A Car mileage

\sphinxlineitem{Return type}
\sphinxAtStartPar
Cars

\end{description}\end{quote}

\end{fulllineitems}


\end{fulllineitems}


\sphinxstepscope


\chapter{Changelogs}
\label{\detokenize{changelogs:changelogs}}\label{\detokenize{changelogs::doc}}\begin{quote}\begin{description}
\sphinxlineitem{v0.0.1}\begin{description}
\sphinxlineitem{New Features}\begin{itemize}
\item {} 
\sphinxAtStartPar
Added all common sphinx templates used in documentation

\item {} 
\sphinxAtStartPar
Added installation, changelog and references

\end{itemize}

\end{description}

\end{description}\end{quote}

\sphinxstepscope


\chapter{References}
\label{\detokenize{references:references}}\label{\detokenize{references::doc}}\begin{itemize}
\item {} \begin{description}
\sphinxlineitem{Understanding sphinx}
\sphinxAtStartPar
\sphinxhref{https://www.sphinx-doc.org/en/master/index.html}{Sphinx}

\end{description}

\item {} \begin{description}
\sphinxlineitem{Understanding how to use the read the docs theme for documentation}
\sphinxAtStartPar
\sphinxhref{https://sphinx-rtd-theme.readthedocs.io/en/stable/installing.html}{Sphinx Read the docs theme}

\end{description}

\item {} \begin{description}
\sphinxlineitem{Full documentation example}
\sphinxAtStartPar
\sphinxhref{https://github.pwc.com/pages/AIA/Concept-Drift-Detection-package/html/index.html}{Drift Detection Documentation}

\end{description}

\item {} \begin{description}
\sphinxlineitem{restructuredtext cheat sheets}
\sphinxAtStartPar
\sphinxhref{https://thomas-cokelaer.info/tutorials/sphinx/rest\_syntax.html}{Reference 1}

\sphinxAtStartPar
\sphinxhref{https://github.com/ralsina/rst-cheatsheet/blob/master/rst-cheatsheet.rst}{Reference 2}

\sphinxAtStartPar
\sphinxhref{https://docutils.sourceforge.io/docs/user/rst/quickref.html}{Reference 3}

\end{description}

\end{itemize}


\renewcommand{\indexname}{Python Module Index}
\begin{sphinxtheindex}
\let\bigletter\sphinxstyleindexlettergroup
\bigletter{m}
\item\relax\sphinxstyleindexentry{module}\sphinxstyleindexpageref{_autosummary/module:\detokenize{module-module}}
\item\relax\sphinxstyleindexentry{module.dummycode}\sphinxstyleindexpageref{_autosummary/module.dummycode:\detokenize{module-module.dummycode}}
\item\relax\sphinxstyleindexentry{module.submodule}\sphinxstyleindexpageref{_autosummary/module.submodule:\detokenize{module-module.submodule}}
\item\relax\sphinxstyleindexentry{module.submodule.dummycode}\sphinxstyleindexpageref{_autosummary/module.submodule.dummycode:\detokenize{module-module.submodule.dummycode}}
\end{sphinxtheindex}

\renewcommand{\indexname}{Index}
\printindex
\end{document}